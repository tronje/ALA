\newcommand{\authorinfotitle}{Jonathan Siems, 6533519, Gruppe 12\\ Jan-Thomas Riemenschneider, 6524390, Gruppe 12 \\ Tronje Krabbe, 6435002, Gruppe 9}
\newcommand{\authorinfo}{Jonathan Siems, Tronje Krabbe, Jan-Thomas Riemenschneider}

\newcommand{\qed}{\square}
\newcommand{\limp}{\underset{n \to \infty}{\lim}}
\newcommand{\limn}{\underset{n \to - \infty}{\lim}}
\newcommand{\todo}{\textbf{\textcolor{red}{TODO}}}
\newcommand{\todaynum}{\the\day.\the\month.\the\year}
\newcommand{\abgabe}{\textbf{\textcolor{red}{ABGABEDATUM}}}
\newcommand{\titleinfo}{ALA BLATTNR. 10 03.07.2014}

\documentclass[a4paper,11pt]{article}
\usepackage[a4paper]{geometry}
\usepackage[german,ngerman]{babel}
\usepackage[utf8]{inputenc}
\usepackage[T1]{fontenc}
\usepackage{amsmath,amssymb,amstext}
\usepackage{mathtools}
\usepackage{enumerate}
\usepackage{breqn}
\usepackage{fancyhdr}
\usepackage{multicol}
\usepackage{color}
\usepackage{microtype}
\usepackage{booktabs}
\usepackage{lmodern}
\usepackage{tikz}
\usepackage{pgfplots}
\usepackage{scrdate}
\usepackage[makeroom]{cancel}
\usetikzlibrary{calc}

\title{\titleinfo}
\author{\authorinfotitle}
\date{\today}

\pagestyle{fancy}
\fancyhf{}
\fancyhead[R]{\authorinfo}
\fancyhead[L]{ALA Hausaufgaben}
\fancyfoot[C]{\thepage}


\begin{document}
\maketitle
    \begin{enumerate}

            \item[\textbf{1.}]
            \begin{enumerate}
            	\item[a)]
            	\begin{align*}
	                &f(x,y,z) = 2x^2+y^2+4z^2-2yx-2x-6y+8 \\[0.4cm]
	                &f_x = 4x -2 \qquad f_{xx} = 4 \\
	                &f_y = 2y -2z -6 \qquad f_{yy} = 2\\
	                &f_z = 8z -2y \qquad f_{zz} = 8 \\[0.4cm]
	                &f_{xy} = 0 \qquad f_{yx} = 0 \\
	                &f_{xz} = 0 \qquad f_{zx} = 0 \\
	                &f_{yz} = -2 \qquad f_{zy} = -2 \\
					&\bordermatrix{
					  &  \cr
					I & 4x -2 \cr
					II & 2y-2z \cr
					III & 88z-2y \cr
					} \\[0.3cm]
	                &x= \frac{1}{2},\quad y = 0,\quad  z = 0 \\
	                \end{align*}
	                Für $f(x,y,z)$ ergibt sich somit $f\left(\frac{1}{2},0,0\right) = \frac{15}{2}$ \\
	                Damit ist der kritische Punkt bei $\left(\frac{1}{2},0,0,\frac{15}{2}\right)$
	                \begin{align*}
	                \text{Hesse Matrix:} \\
	                &A = \begin{pmatrix}
	                	 4 & 0 & 0 \\
	                	 0 & 2 & -2 \\
	                	 0 & 0 & 8 \\
	                \end{pmatrix} \\ \\ 
	                &4 > 0, \Delta = 64 \qquad \text{A ist somit positiv definit.}
            	\end{align*}

            \end{enumerate}
            \item[\textbf{2.}]
            \begin{enumerate}
            	\item[a)]
            		\begin{enumerate}
            			\item[(i)]
            			
            			     $\bold{x^2+2x-35 = 0}$ \\
            			     \begin{align*}
            			     x_{1|2} &= -\frac{2}{2} \pm \sqrt{\left(\frac{2}{2}\right)^2+35} \\
            			     &= -1 \pm \sqrt{1+35} \\[0.4cm]
            			     &\bold{x_1 = 5 \qquad x_2 = -7} \\
            			     \end{align*}
            			     
            			     \item[(ii)]

            			     $\bold{x^2+2x+10 = 10}$ \\
            			     \begin{align*}
            			     x_{1|2} &= - \frac{2}{2} \pm \sqrt{\left(\frac{2}{2}\right)^2 -10} \\
            			     &= -1 \pm \sqrt{1-10} \\
            			     &\bold{x_1 = -1+3i \qquad x_2 = -1 -3i}
            			     \end{align*}
            			 	
            			 	\item[(iii)] 
            			 
            			 $\bold{x^2 -18x + 81 = 0} $\\
            			 \begin{align*}
            			 x_{1|2} &= - \frac{18}{2} \pm \sqrt{\left(-\frac{18}{2}\right)^2 -81} \\
            			 &= -9 \pm \sqrt{\left(-9\right)^2 -81} \\
            			 &\bold{x_1 = 9}
            			 \end{align*}
            		\end{enumerate}
            \end{enumerate}


            \item[\textbf{3.}]


            \item[\textbf{4.}]


    \end{enumerate}

\end{document}
