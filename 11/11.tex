\newcommand{\authorinfotitle}{Jonathan Siems, 6533519, Gruppe 12\\ Jan-Thomas Riemenschneider, 6524390, Gruppe 12 \\ Tronje Krabbe, 6435002, Gruppe 9}
\newcommand{\authorinfo}{Jonathan Siems, Tronje Krabbe, Jan-Thomas Riemenschneider}

\newcommand{\qed}{\square}
\newcommand{\limp}{\underset{n \to \infty}{\lim}}
\newcommand{\limn}{\underset{n \to - \infty}{\lim}}
\newcommand{\todo}{\textbf{\textcolor{red}{TODO}}}
\newcommand{\todaynum}{\the\day.\the\month.\the\year}
\newcommand{\abgabe}{\textbf{\textcolor{red}{ABGABEDATUM}}}
\newcommand{\titleinfo}{ALA BLATTNR. 11 10.07.2014}

\documentclass[a4paper,11pt]{article}
\usepackage[a4paper]{geometry}
\usepackage[german,ngerman]{babel}
\usepackage[utf8]{inputenc}
\usepackage[T1]{fontenc}
\usepackage{amsmath,amssymb,amstext}
\usepackage{mathtools}
\usepackage{enumerate}
\usepackage{breqn}
\usepackage{fancyhdr}
\usepackage{multicol}
\usepackage{color}
\usepackage{microtype}
\usepackage{booktabs}
\usepackage{lmodern}
\usepackage{tikz}
\usepackage{pgfplots}
\usepackage{scrdate}
\usepackage[makeroom]{cancel}
\usetikzlibrary{calc}

\title{\titleinfo}
\author{\authorinfotitle}
\date{\today}

\pagestyle{fancy}
\fancyhf{}
\fancyhead[R]{\authorinfo}
\fancyhead[L]{ALA Hausaufgaben}
\fancyfoot[C]{\thepage}


\begin{document}
\maketitle
    \begin{enumerate}

            \item[\textbf{1.}]

            \item[\textbf{2.}]


            \item[\textbf{3.}]
                \begin{enumerate}
                    \item[a)]
                        \begin{itemize}
                            \item $f_3(n)$\\
                            \item $f_2(n)$\\
                            \item $f_6(n)$\\
                            \item $f_1(n)$\\
                            \item $f_4(n)$\\
                            \item $f_5(n)$\\
                        \end{itemize}
                        $f_3$ wächst am langsamsten, da es sich hier um lineares
                        Wachstum handelt. Danach kommt $f_2$, was ein Polynom ist.
                        Diese wächt allerdings langsamer als die anderen Polynome,
                        da es sich um den Sonderfall der Wurzelfunktion handelt.
                        Als nächstes könnte man denken käme $f_1$, da $f_6$ ja
                        schneller wachsen sollte, weil es ein Polynom multipliziert
                        mit einem Logarithmus ist. Diese Annahme ist falsch,
                        da bei der Landaunotation stets nur auf die mächtigste
                        Komponente geachtet wird, was in dem Fall $n^2$ wäre, und
                        dies wächst langsamer als $n^{2.5}$. Zuletzt kommen
                        die beiden Exponentialfunktionen $f_4$ und $f_5$.
                        Eigentlich teilen diese beiden sich den Platz für das
                        schnellste Wachstum, da die Basis von $10$ bzw. $100$
                        in der Landaunotation keinen Unterschied macht.
                    \item[b)]
                        \begin{itemize}
                            \item $g_3(n)$\\
                            \item $g_4(n)$\\
                            \item $g_1(n)$\\
                            \item $g_5(n)$\\
                            \item $g_2(n)$\\
                            \item $g_7(n)$\\
                            \item $g_6(n)$\\
                        \end{itemize}
                        Die Liste beginnt mit den Funktionen, die kein
                        $n$ im Exponenten haben, da diese immer am langsamsten
                        wachsen. Dabei wächst $g_4$ schneller als $g_3$, da
                        es sich hierbei um ein Polynom handelt. Danach kommt
                        $g_1$, da hier ein $n$ im Exponenten zu finden ist,
                        der Exponent jedoch relativ langsam wächst, aufgrund
                        des Logarithmus und der Wurzel. Als nächstes kommt 
                        $g_5$, was überraschend scheinen kann, da es sowohl
                        in der Basis als auch im Exponenten ein $n$ hat. Jedoch
                        wächst der Exponent nur logarithmisch, also langsamer
                        als die Exponenten der übrigen Funktionen. Deshalb 
                        wird $g_5$ unter den restlichen Funktionen eingeordnet,
                        da die Basis nicht ausschlaggebend genug ist.
                        Es folgen die Funktionen mit Basis 2, die nach 
                        ihren Exponenten geordnet sind. Die Begründung für diese
                        Ordnung ist trivial und z.T. bereits in a) geschehen.
                \end{enumerate}

            \item[\textbf{4.}]
                \begin{enumerate}
                    \item[a)]
                        (i): falsch, da ein Polynom nicht genauso schnell wächst
                        wie eine Exponentialfunktion.\\
                        (ii): richtig, da $3^n$ mindestens so schnell wächst wie
                        $2^n$. Der Beweis ist trivial.\\
                        (iii): richtig, da der Unterschied in der Basis für 
                        besonders hohe Werte keinen Unterschied macht.\\
                        (iv): richtig, denn wegen des Exponenten $\sqrt{n}$
                        wächst die Funktion auf der linken Seite langsamer.\\
                        (v): falsch, $\ln(n)$ wächst wesentlich schneller als
                        $\ln(\sqrt{n})$.\\
                        (vi): falsch, der Logarithmus wächst wesentlich
                        schneller als die Wurzel eines Logarithmus.
                    \item[b)]
                        $f(n)=n$, $g(n)=n^{1+\sin(n)}$
                \end{enumerate}

    \end{enumerate}

\end{document}
