\newcommand{\authorinfotitle}{Jonathan Siems, 6533519, Gruppe 12\\ Jan-Thomas Riemenschneider, 6524390, Gruppe 12 \\ Tronje Krabbe, 6435002, Gruppe 9}
\newcommand{\authorinfo}{Jonathan Siems, Tronje Krabbe, Jan-Thomas Riemenschneider}

\newcommand{\bra}[1]{\left(#1\right)}
\newcommand{\qed}{\square}
\newcommand{\limp}{\underset{n \to \infty}{\lim}}
\newcommand{\limn}{\underset{n \to - \infty}{\lim}}
\newcommand{\todo}{\textbf{\textcolor{red}{TODO}}}
\newcommand{\todaynum}{\the\day.\the\month.\the\year}
\newcommand{\abgabe}{\textbf{\textcolor{red}{ABGABEDATUM}}}
\newcommand{\titleinfo}{ALA BLATTNR. 05 08.05.2014}

\documentclass[a4paper,11pt]{article}
\usepackage[a4paper]{geometry}
\usepackage[german,ngerman]{babel}
\usepackage[utf8]{inputenc}
\usepackage[T1]{fontenc}
\usepackage{amsmath,amssymb,amstext}
\usepackage{mathtools}
\usepackage{enumerate}
\usepackage{breqn}
\usepackage{fancyhdr}
\usepackage{multicol}
\usepackage{color}
\usepackage{microtype}
\usepackage{booktabs}
\usepackage{lmodern}
\usepackage{tikz}
\usepackage{pgfplots}
\usetikzlibrary{calc}

\title{\titleinfo}
\author{\authorinfotitle}
\date{\today}

\pagestyle{fancy}
\fancyhf{}
\fancyhead[R]{\authorinfo}
\fancyhead[L]{ALA Hausaufgaben}
\fancyfoot[C]{\thepage}


\begin{document}
\maketitle
    \begin{enumerate}
        % Aufgabe 1
        \item[\textbf{1.}]
            \todo
              
        % Aufgabe 2
        \item[\textbf{2.}]
        	\[
            \lim\limits_{x \rightarrow 6}\bra{\frac{f(x) - f(6)}{x-6}} =
            \lim\limits_{x \rightarrow 6}\bra{\frac{|3 - \frac{1}{2}x| - |3 - \frac{1}{2} \cdot 6}{x-6}} =
            \lim\limits_{x \rightarrow 6}\bra{\frac{|3 - \frac{1}{2}x|}{x-6}} =
            \]
            \[
            \lim\limits_{x \rightarrow 6}\bra{\sqrt{\frac{(3 - \frac{1}{2}x)^2}{(x-6)^2}}} =
            \sqrt{\lim\limits_{x \rightarrow 6}\bra{\frac{9 - 3x + \frac{1}{4}x^2}{x^2 - 2x + 36}}} =
            \sqrt{\frac{1}{4}} = \frac{1}{2}
            \]

        \begin{center}\begin{tikzpicture}[>=stealth]
            \begin{axis}[
                ymin=0,ymax=4,
                xmin=-1,xmax=8,
                x=1cm, y=1cm,
                axis x line=middle,
                axis y line=middle,
                axis line style=->,
                xlabel={$x$},
                ylabel={$y$},
                ]
                \addplot[no marks, black, -] expression[domain=-1:6,samples=4]{3-(1/2)*x} node[pos=0.65,anchor=north]{};
                \addplot[no marks, black, -] expression[domain=6:9,samples=4]{-3+(1/2)*x} node[pos=0.65,anchor=north]{};
            \end{axis}
        \end{tikzpicture}\end{center}
        % Aufgabe 3
        \item[\textbf{3.}]
            \begin{enumerate}
            \item[{a)}]
                    \begin{align*}
                    \text{Umformen:}& \\
                    f(x) \ &= \ (x\ +\ 1)^{x+2} \ = \ e^{ln(x+1)^{x+2}} \ = \  e^{ln(x+1)(x+2)} \\
                    \text{Differenzieren:}& \\
                    f'(x) \ &= \  \left(e^{ln(x+1)(x+2)} \right)'  \\
                    &=  e^{ln(x+1)(x+2)} \cdot \left((x+2) \cdot ln(x+1)\right)'  \\
                    &= e^{ln(x+1)(x+2)} \cdot \left(x \cdot ln(x+1) \ +\ (x+2)\cdot \frac{1}{(x+1)}\right) \\
                    &= (x\ +\ 1)^{x+2} \cdot \left(x \cdot ln(x+1) \ +\ \frac{x+2}{(x+1)}\right) 
                    \end{align*}
             \item[{b)}]
                    \begin{enumerate}

                    \item[(i)]
                        \begin{align*}
                            \text{Umformen:}& \\
                            g(x) \ &= \ (x^2 +5)^{x^4+3} = e^{ln(x^2+5)^{x^4+3}} = e^{(x^4 +3) \cdot ln(x^2+5) } \\ 
                            \text{Differenzieren:}& \\
                            g'(x) &= \left(e^{(x^4 +3) \cdot ln(x^2+5)} \right)' \\
                            &= e^{(x^4 +3) \cdot ln(x^2+5)} \cdot \left((x^4 +3) \cdot ln(x^2+5) \right) \\
                            &= e^{(x^4 +3) \cdot ln(x^2+5)} \cdot \left(4x^3 \cdot  ln(x^2+5) + (x^4 +3) \cdot \frac{2x}{x^2+5} \right)\\
                            &= (x^2 +5)^{x^4+3} \cdot \left(4x^3 \cdot  ln(x^2+5) + \frac{2x^5 + 6x}{x^2+5} \right)
                        \end{align*}
                    \item[(ii)]
                        \begin{align*}
                            \text{Umformen:} & \\
                            h(x) &= (x^4 +3)^{\sqrt{3x+1}} = e^{ln(x^4 +3)^{\sqrt{3x+1}}} = e^{\sqrt{3x+1}\cdot ln(x^4+3)} \\
                            \text{Differenzieren:}& \\
                            h'(x) &= e^{\sqrt{3x+1}\cdot ln(x^4+3)} \\
                            h'(x) &=  e^{\sqrt{3x+1}\cdot ln(x^4+3)} \cdot \left(\sqrt{3x+1}\cdot ln(x^4+3) \right) \\
                            &= e^{\sqrt{3x+1}\cdot ln(x^4+3)} \cdot \left((\sqrt{3x+1})' \cdot ln(x^4+3) + \sqrt{3x+1} \cdot \frac{4x^3}{x^4+3} \right) \\
                            &= (x^4 +3)^{\sqrt{3x+1}} \cdot \left( \frac{3}{2\sqrt{1+ 3 \ x}} \cdot ln(x^4+3) +  \frac{4x^3\sqrt{3x+1}}{x^4+3} \right)
                        \end{align*}
                    \end{enumerate}
            \item[{c)}] 
                        \begin{align*}
                            \text{Umformen:} & \\
                            f(x) &= 3^x \ = \ e^{ln(3^x)} \ = \  e^{x \cdot ln3} \\
                            \text{Differenzieren:}& \\
                            f'(x) &= \left(e^{x \cdot ln3} \right)' \\
                            &= e^{x \cdot ln3} \cdot \left(x \cdot ln3 \right)'\\
                            &= e^{x \cdot ln3} \cdot \left( 1* ln3+x \cdot \frac{0}{3}\right) \\ 
                            &= 3^x  \cdot \left(ln3 \right) \\
                            \\
                            \text{Umformen:} & \\
                            f(x) &= x^{\frac{1}{3}} = e^{ln(x^{\frac{1}{3}})} = e^{\frac{1}{3}\cdot ln\ x} \\
                            \text{Differenzieren:}& \\
                            f'(x) &= \left(e^{\frac{1}{3}\cdot lnx}\right)' \\
                            f'(x) &= e^{\frac{1}{3}\cdot ln\ x} \cdot \left(\frac{1}{3}\cdot ln x\right)' \\
                            &= e^{\frac{1}{3}\cdot ln x} \cdot \left( ln x + \frac{1}{3}  \cdot \frac{1}{x} \right) \\
                            &= x^{\frac{1}{3}} \cdot \left(ln x + \frac{\frac{1}{3}x}{x} \right) \\ \\ \\
                        \end{align*}             
            \end{enumerate}

        % Aufgabe 4
        \item[\textbf{4.}]
            \begin{enumerate}
                \item[(ii)] 
                   $ f'(x)= cos(x^2) \cdot 2x$
                \item[(iii)]
                    $f'(x)= 2 \cdot sin(x) \cdot cos(x)$
                \item[(iv)]
                    $f'(x) = cos(2x)$
                \item[(v)]
                    $f'(x) = \frac{1}{2\sqrt{-(-1+x)x}}$
                \item[(vi)]
                    $ f'(x)= (x^3 - 1)^{arctan(x)} \cdot \frac{1}{1+x^2} \cdot ln(x^3 - 1) + arctan(x) \cdot \left( \frac{1}{x^3-1} \cdot 3x^2 \right) $    
            \end{enumerate} 

        % Aufgabe 5           
        \item[\textbf{5.}]
            \todo

        % Aufgabe 6
        \item[\textbf{6.}]
            \begin{enumerate}
                \item[a)]
                    \begin{align*}
                        g'(p)=10^5 \left( - \frac{1}{p^2} + \frac{6}{p^3} \right) = 10^5 \frac{6-p}{p^3}
                    \end{align*}
                    Einsetzen der 0:
                    \begin{align*}
                        g'(p)=0 \Leftrightarrow p=6
                    \end{align*}
                    Die m"oglichen globalen Maxima liegen demnach bei $ p=3, p=6, p=100 $. Jetzt muss nur noch eingesetzt werden:
                    \begin{align*}
                        g(3)=0 \\
                        g(6)=10^5 \frac{1}{12} \\
                        g(100)=10^5 \frac{99}{10000}
                    \end{align*}
                    Also gilt: $ g(6) > g(3) $ und $g(6)>g(100)$. Das bedeutet, dass das globale Maximum bei $p=6$ liegt.
                    Demnach ist der Gewinn bei einem Preis von 6 Euro maximal.
                \item[b)]
                    \begin{enumerate}
                        \item[(i)]
                            \[
                                f'(x)=21x^6 + 25x^4 + 6x^2 + 1 > 0
                            \]
                            Die Funktion ist streng monoton steigend, d.h. das globale Maximum liegt bei $x=10$ und das globale Minimum bei
                            $x=-10$.
                        \item[(ii)]
                            \[
                                g'(x)=2e^{2x-1}-e^{x+1}=0 \Leftrightarrow 2e^{2x-1} = e^{x+1} \Leftrightarrow x=2- \log(2)
                            \]
                            Kandidaten f"ur globale Extrema sind also $x=-2, x=2, x=2-log(2)$. Einsetzen ergibt:
                            Das Maximum liegt bei $x=2$, das Minimum bei $x=2-log(2)$.
                        \item[(iii)]

                    \end{enumerate}
            \end{enumerate}
    \end{enumerate}


\end{document}

