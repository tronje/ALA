\newcommand{\authorinfotitle}{Jonathan Siems, 6533519, Gruppe 12\\ Jan-Thomas Riemenschneider, 6524390, Gruppe 12 \\ Tronje Krabbe, 6435002, Gruppe 9}
\newcommand{\authorinfo}{Jonathan Siems, Tronje Krabbe, Jan-Thomas Riemenschneider}

\newcommand{\qed}{\square}
\newcommand{\limp}{\underset{n \to \infty}{\lim}}
\newcommand{\limn}{\underset{n \to - \infty}{\lim}}
\newcommand{\todo}{\textbf{\textcolor{red}{TODO}}}
\newcommand{\todaynum}{\the\day.\the\month.\the\year}
\newcommand{\abgabe}{\textbf{\textcolor{red}{15.05.2014}}}
\newcommand{\titleinfo}{ALA 05 15.05.2014}

\documentclass[a4paper,11pt]{article}
\usepackage[a4paper]{geometry}
\usepackage[german,ngerman]{babel}
\usepackage[utf8]{inputenc}
\usepackage[T1]{fontenc}
\usepackage{amsmath,amssymb,amstext}
\usepackage{mathtools}
\usepackage{enumerate}
\usepackage{breqn}
\usepackage{fancyhdr}
\usepackage{multicol}
\usepackage{color}
\usepackage{microtype}
\usepackage{booktabs}
\usepackage{lmodern}
\usepackage{tikz}
\usepackage{pgfplots}
\usepackage{scrdate}
\usetikzlibrary{calc}

\title{\titleinfo}
\author{\authorinfotitle}
\date{\today}

\pagestyle{fancy}
\fancyhf{}
\fancyhead[R]{\authorinfo}
\fancyhead[L]{ALA Hausaufgaben}
\fancyfoot[C]{\thepage}


\begin{document}
\maketitle
    \begin{enumerate}
        % Aufgabe 1
        \item[\textbf{1.}]
            \begin{align*}
                f(x)&=7x^3-42x^2+63x-2\\
                f'(x)&=21x^2-84x+63\\
                f''(x)&=42x-84\\
                F(x)&= \frac{7}{4}x^4-\frac{42}{3}x^3+\frac{63}{2}x^2-2x
            \end{align*}
            Globale Extrema von $f$ finden wir an Stellen, f"ur die gilt: $f'(x)=0$.
            \begin{align*}
                f'(x)=0 \\
                \Leftrightarrow 0=21x^2-84x+63 \\
                \Leftrightarrow x^2-\frac{84}{21}x+\frac{63}{21}=0 \\
                \Rightarrow x_{1,2}=\frac{84}{42} \pm \sqrt{\left(- \frac{84}{42}\right)^2-\frac{63}{21}} \\
                \Rightarrow x = 1 \vee x=3
            \end{align*}
            Die Extrema liegen also bei $x=1$ und $x=3$. Einsetzen in die zweite Ableitung:
            \begin{align*}
                f''(1)=42 \cdot 1 - 84=-42 \\
                f''(3)=42 \cdot 3 - 84=42
            \end{align*}
            Das Maximum, also die H"ochsttemperatur liegt also an der Stelle 1, das Minimum, also die Teifsttemperatur, an der Stelle 3.\\
            Die Berechnung des Durchschnittswertes erfolgt folgendermaßen: errechne das Integral von $f$ im Interval [1,3] und teile es durch die
            Intervallänge, also 2:
            \begin{align*}
                &&\frac{\int_1^3 f(x) \mathrm{d}x}{2} \\
                &=& \frac{F(3)-F(1)}{2} \\
                &=& \frac{(\frac{567}{4}-\frac{1134}{3}+\frac{567}{2}-6)-(\frac{7}{4}-\frac{42}{3}+\frac{63}{2}-2)}{2} \\
                &=& \frac{24}{2}=12
            \end{align*}
            Die Tagesdurchschnittstemperatur ist also $12^{\circ}C$.
        % Aufgabe 2
        \item[\textbf{2.}]
            \subitem(i)
                \begin{align*}
                    &&\int_1^3 x^2-x-6 \mathrm{d}x \\
                    &=&[\frac{1}{3}x^3-\frac{1}{2}x^2-6x]_1^3\\
                    &=&(9-\frac{9}{2}-18)-(\frac{1}{3}-\frac{1}{2}-6)\\
                    &=&-\frac{22}{3}
                \end{align*}
            \subitem(ii)
                \begin{align*}
                    &&\int_1^3 x^{\frac{1}{3}} \mathrm{d}x \\
                    &=&[\frac{3}{4}x^{\frac{4}{3}}]_1^3 \\
                    &=&\frac{3}{4}3^{\frac{4}{3}} - \frac{3}{4}1^{\frac{4}{3}}
                \end{align*}
            \subitem(iii)
                \begin{align*}
                    &&\int_1^3 \frac{1}{x^2} \mathrm{d}x \\
                    &=&[\tan^{-1}(x)]_1^3 \\
                    &=&\tan^{-1}(3) - \tan^{-1}(1)
                \end{align*}
            \subitem(iv)
                \begin{align*}
                    &&\int_1^3 ln x \mathrm{d}x \\
                    &=&[x \cdot (ln(x)-1)]_1^3 \\
                    &=&1+3(ln(3)-1)
                \end{align*}
            \subitem(v)
                \begin{align*}
                    &&\int_1^3 e^{-x} \mathrm{d}x \\
                    &=&[-e^{-x}]_1^3 \\
                    &=&(-e^{-3})-(-e^{-1})\\
                \end{align*}

        % Aufgabe 3
        \item[\textbf{3.}]
	        \subitem (i)
                \begin{align*}
                    &&\int x^4+2x^3-x+5 \mathrm{d}x \\
                    &=&\frac{1}{5}x^5+\frac{2}{4}x^4-\frac{1}{2}x^2+5x
                \end{align*}
            \subitem (ii)
                \begin{align*}
                    \int x^{-\frac{3}{2}} \mathrm{d}x \\
                    = 2x^{-\frac{1}{2}}
                \end{align*}
            \subitem (iii)
                \begin{align*}
                     \mathrm{d}x
                \end{align*}
            \subitem (iv)
                \begin{align*}
                 \mathrm{d}x
                \end{align*}
            \subitem (v)
                \begin{align*}
                 \mathrm{d}x
                \end{align*}
        % Aufgabe 4
        \item[\textbf{4.}]
            Zunächst überprüfen wir, ob alle Voraussetzungen zur Ausführung des Newton-Verfahrens erfüllt sind:\\
            (1) $f'(x)=3x^2+4x+10 \neq 0$ für $x \in [1,2]$\\
            (2) $f''(x)$ ist ein Polynom, also im gegebenen Interval ganz vorhanden und stetig. \\
            (3) $f(a) \cdot f(b) < 0 \Leftrightarrow -13 \cdot 16 < 0$. \\
            Nun können wir das eigentliche Verfahren anwenden.
            \begin{align*}
                x_{n+1}=x_n - \frac{f(x_n)}{f'(x_n)}\\
                x_0 = 1 \\
                \Rightarrow x_1 = 1- \frac{-13}{17} = 1.764705...\\
                \Rightarrow x_2 = x_1 - \frac{f(x_1)}{f'(x_1)} = 1.409760...\\
                \Rightarrow x_3 = x_2 - \frac{f(x_2)}{f'(x_2)} = 1.369288...\\
                \Rightarrow x_4 = x_3 - \frac{f(x_3)}{f'(x_3)} = 1.368808...
            \end{align*}
            Die Nullstelle liegt also in der Nähe von $1.368808...$
    \end{enumerate}


\end{document}