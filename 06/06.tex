\newcommand{\authorinfotitle}{Jonathan Siems, 6533519, Gruppe 12\\ Jan-Thomas Riemenschneider, 6524390, Gruppe 12 \\ Tronje Krabbe, 6435002, Gruppe 9}
\newcommand{\authorinfo}{Jonathan Siems, Tronje Krabbe, Jan-Thomas Riemenschneider}

\newcommand{\qed}{\square}
\newcommand{\limp}{\underset{n \to \infty}{\lim}}
\newcommand{\limn}{\underset{n \to - \infty}{\lim}}
\newcommand{\todo}{\textbf{\textcolor{red}{TODO}}}
\newcommand{\todaynum}{\the\day.\the\month.\the\year}
\newcommand{\abgabe}{\textbf{\textcolor{red}{ABGABEDATUM}}}
\newcommand{\titleinfo}{ALA BLATTNR. 06 22.05.2014}

\documentclass[a4paper,11pt]{article}
\usepackage[a4paper]{geometry}
\usepackage[german,ngerman]{babel}
\usepackage[utf8]{inputenc}
\usepackage[T1]{fontenc}
\usepackage{amsmath,amssymb,amstext}
\usepackage{mathtools}
\usepackage{enumerate}
\usepackage{breqn}
\usepackage{fancyhdr}
\usepackage{multicol}
\usepackage{color}
\usepackage{microtype}
\usepackage{booktabs}
\usepackage{lmodern}
\usepackage{tikz}
\usepackage{pgfplots}
\usepackage{scrdate}
\usetikzlibrary{calc}

\title{\titleinfo}
\author{\authorinfotitle}
\date{\today}

\pagestyle{fancy}
\fancyhf{}
\fancyhead[R]{\authorinfo}
\fancyhead[L]{ALA Hausaufgaben}
\fancyfoot[C]{\thepage}


\begin{document}
\maketitle
    \begin{enumerate}
        % Aufgabe 1
        \item[\textbf{1.}]
         \begin{enumerate}
             \item[(i)]
            $$ \int sin \left( \sqrt{3x+7}\right)dx$$$$ \ \Downarrow $$$$ \int f(x)dx = \frac{2 \cdot (sin (\sqrt{3x+7})-\sqrt{3x+7} \cdot cos(\sqrt{3x+7}))}{3} + C $$
            \item[(ii)]
            $$ \int cos \left( \sqrt[3]{x}\right)dx $$$$ \ \Downarrow \ $$$$  \int f(x)dx = 3\left(x^{\frac{2}{3}}-2\right) sin\left(\sqrt[3]{x}\right) +6\sqrt[3]{x} cos\left(\sqrt[3]{x}\right) + C $$
            \item[(iii)]
            $$ \int e^{\sqrt{5x+3}}dx $$$$ \Downarrow \ $$$$ \int f(x)dx = \frac{2}{5}e^{\sqrt{5x+3}}(\sqrt{5x+3}-1) + C $$
            \item[(iv)]
            $$ \int ln (4x+3)dx \left(\text{ für } x > - \frac{3}{4}\right)  $$$$\ \Downarrow \ $$$$ \int f(x)dx = \frac{(4x+3) \cdot ln(4x+3) -4x -3}{4} + C $$

         \end{enumerate}
              
        % Aufgabe 2
        \item[\textbf{2.}]
        \begin{enumerate}
            \item[(i)]
             $$\int \frac{x+1}{x^2-x-6}dx $$$$ \Downarrow $$$$  \int f(x)dx = \frac{1}{5}(4log(3-x)+log(x+2)) +C $$
             \item[(ii)]
             $$ \int \frac{2x+1}{x^2-4x+4}dx $$$$ \Downarrow $$$$  \int f(x)dx = 2 \cdot log(x-2) - \frac{5}{x-2} + C $$
             \item[(iii)] 
             $$ \int \frac{4x+1}{x^2+4x+8}dx  $$$$ \Downarrow $$$$  \int f(x)dx = 2 \cdot log(x^2+4x+8) - \frac{7}{2}tan^{-1}\left(\frac{x+2}{2}\right) +C $$
 
        \end{enumerate}



        % Aufgabe 3
        \item[\textbf{3.}]
         
            \begin{enumerate}
                \item[a)]
                    \(f(1)=9 \cdot e^{-\frac{1}{3}} \approx 6,45\)\\
                    \(f(2)=18 \cdot e^{-\frac{2}{3}} \approx 9,24\)\\
                    \(f(6)=54 \cdot e^{-2} \approx 7,31\)\\
                    \(f(12)=108 \cdot e^{-4} \approx 1,98\)\\
                    \(f(24)=216 \cdot e^{-8} \approx 0,07\)\\
                \item[b)]
                    \( f'(t)= 9t \cdot e^{-\frac{1}{3}t} \cdot -\frac{1}{3} + 9e^{-\frac{1}{3}t} = 9e^{-\frac{1}{3}t} - 3e^{-\frac{1}{3}t} \cdot t
                    = 0 \Leftrightarrow 9e^{-\frac{1}{3}t} = 3e^{-\frac{1}{3}t} \cdot t \Leftrightarrow t=3 \)\\
                    \( f'(3)=27e^{-1} \approx 9,93 \)\\
                    Die maximale Konzentration wird also nach 3 Stunden erreicht und beträgt circa 9,93 Milligramm pro Liter.
                \item[c)]
                    Hierfür berechnen wir das Integral von f zwischen 0 und 6, und teilen das Ergebnis durch $6-0=6$.\\
                    \begin{align*}
                        \int_0^6 9t \cdot e^{-\frac{1}{3}t} \mathrm{d}t\\
                        = [-27 e^{-\frac{1}{3}t} \cdot (t+3)]_0^6\\
                        = -243 e^{-2} + 81\\
                        = 48,11... \\
                        \frac{48,11...}{6} \approx 8,02
                    \end{align*}
                    Die durchschnittliche Konzentration in den ersten 6 Stunden liegt demnach bei circa 8,02 Milligramm pro Liter.\\
                    Als, wenn auch nur sehr grobe, Probe kann man die ersten drei Werte aus Aufgabenteil a addieren und durch drei Teilen,
                    was $7, \overline{6}$ ergibt.
                \item[d)]
                    Analog zu Aufgabenteil c):\\
                    \begin{align*}
                        \int_6^12 9t \cdot e^{-\frac{1}{3}t} \mathrm{d}t\\
                        = [-27 e^{-\frac{1}{3}t} \cdot (t+3)]_6^12\\
                        = -405 e^{-4} + 243 e^{-2} \\
                        = 25,46...\\
                        \frac{25,46...}{6} \approx 4,24
                    \end{align*}
                    In den zweiten 6 Stunden liegt die Durchschnittliche Konzentration also bei 4,24 mg/l.\\
                    Auch hier kann wieder, als grobe Probe, Aufgabenteil a herangezogen werden.\\
                    $\frac{7,31+1,98}{2} \approx 4,65$.
                \item[e)]
                    Skizze:\\
                    \begin{tikzpicture}[scale=0.9]
                        \begin{axis}[
                            ymin=0,ymax=11,
                            xmin=0,xmax=14.5,
                            x=1cm, y=0.5cm,
                            axis x line=middle,
                            axis y line=middle,
                            axis line style=->,
                            xlabel={$t$},
                            ylabel={$f(t)$},
                            ]
                            \addplot[very thick, no marks, black, -] expression[domain=0:12,samples=100]{9*x*e^(-(1/3)*x)};
                            \addplot[very thick, no marks, black, -, dotted] expression[domain=12:14,samples=20]{9*x*e^(-(1/3)*x)};
                        \end{axis}

                        \node[fill, inner sep=2pt, circle, label=above right:{\small Stärkster Abbau/Wendepunkt}] at (6,3.66) {};
                    \end{tikzpicture}
                    \( f'(t)=9e^{-\frac{1}{3}t} - 3e^{-\frac{1}{3}t} \cdot t \)\\
                    \( f''(t)= e^{-\frac{1}{3}t} \cdot x - 6e^{-\frac{1}{3}t} = 0
                    \Leftrightarrow e^{-\frac{1}{3}t} \cdot x = 6e^{-\frac{1}{3}t} \Leftrightarrow x=6 \)
            \end{enumerate}

        % Aufgabe 4
        \item[\textbf{4.}]
	\todo

    \end{enumerate}


\end{document}

