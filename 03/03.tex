\newcommand{\authorinfotitle}{Jonathan Siems, 6533519, Gruppe 12\\ Tronje Krabbe, 6435002, Gruppe 9}
\newcommand{\authorinfo}{Jonathan Siems, Tronje Krabbe}

\newcommand{\titleinfo}{ALA 03 24.04.2014}
\newcommand{\qed}{\square}
\newcommand{\limp}{\underset{n \to \infty}{\lim}}
\newcommand{\limn}{\underset{n \to - \infty}{\lim}}
\newcommand{\limo}{\underset{n \to 0}{\lim}}
\newcommand{\todo}{\textbf{\textcolor{red}{TODO}}}
\newcommand{\todaynum}{\the\day.\the\month.\the\year}
\newcommand{\brac}[1]{\left(#1\right)}

\documentclass[a4paper,12pt]{article}
%\usepackage[a4paper]{geometry}
\usepackage[german,ngerman]{babel}
\usepackage[utf8]{inputenc}
\usepackage[T1]{fontenc}
\usepackage{amsmath,amssymb,amstext}
\usepackage{mathtools}
\usepackage{enumerate}
\usepackage{breqn}
\usepackage{fancyhdr}
\usepackage{multicol}
\usepackage{color}
\usepackage{microtype}
\usepackage{booktabs}
\usepackage{tikz}
\usepackage{pgfplots}
\usepackage{lmodern}

\title{\titleinfo}
\author{\authorinfotitle}
\date{\today}

\pagestyle{fancy}
\fancyhf{}
\fancyhead[R]{\authorinfo}
\fancyhead[L]{ALA Hausaufgaben}
\fancyfoot[C]{\thepage}


\begin{document}
\maketitle
	\begin{enumerate}
	% Aufgabe 1
        \item[\textbf{1.}]
            \begin{enumerate}
                \item[a)]
                \[
                    \begin{tikzpicture}
                        \begin{axis}[xlabel=x,ylabel=f(x)]
                            \addplot[color=black] coordinates {
                                (0,2)
                                (1,3.5)
                                (2,5)
                            };
                            \addplot[color=black] coordinates {
                                (2,3)
                                (3,2)
                                (4,1)
                            };
                            \addplot[color=black] coordinates {
                                (4,1)
                                (5,1.5)
                                (6,2)
                            };
                            \addplot[color=black] coordinates {
                                (6,3)
                                (7,4)
                                (8,5)
                            };
                            \addplot[color=black] coordinates {
                                (8,5)
                                (9,7)
                                (10,9)
                            };
                        \end{axis}
                    \end{tikzpicture}
                \]
                \\
                Die Unstetigkeitsstellen von $f$ befinden sich bei $x=2$ und $x=6$.
                \item[b)]
                \[
                    \begin{tikzpicture}
                        \begin{axis}[xlabel=x,ylabel=g(x),height=5cm,width=10cm]
                            \addplot[color=black] coordinates {
                                (-3,0)
                                (-2,1)
                            };
                            \addplot[color=black] coordinates {
                                (-2,0)
                                (-1,1)
                            };
                            \addplot[color=black] coordinates {
                                (-1,0)
                                (0,1)
                            };
                            \addplot[color=black] coordinates {
                                (0,0)
                                (1,1)
                            };
                            \addplot[color=black] coordinates {
                                (1,0)
                                (2,1)
                            };
                            \addplot[color=black] coordinates {
                                (2,0)
                                (3,1)
                            };
                        \end{axis}
                    \end{tikzpicture}
                \]
                \\
                Sei $x_0 \in D(g)\textbackslash \mathbb{Z}$.
                So muss für jede Folge $(x_n)_{n \in \mathbb{N}}$ mit
                \[
                    \limp x_n = x_0
                \]
                gelten:
                \[
                    \limp g(x_n) = g(x_0) % = x_0 - \lfloor x_0 \rfloor
                \]
                Also:
                \[
                    \limp g(x_n)
                    = \limp (x_n - \lfloor x_n \rfloor)
                    = \limp x_n - \limp \lfloor x_n \rfloor
                    = x_0 - \limp \lfloor x_n \rfloor
                \]
                Da Abrundung nicht stetig ist, weiss ich leider nicht weiter...
                % So muss gelten: Es gibt für alle $\epsilon > 0$ ein $\delta > 0$,
                % so dass $|g(x) - g(x_0)|< \epsilon$ für alle
                % $x \in D(g)$ mit $|x-x_0| < \delta$ gilt.
            \end{enumerate}
        % Aufgabe 2
        \item[\textbf{2.}]
            \begin{enumerate}
                \item[a)]
                    \begin{align*}
                        &&\limp \left( \frac{\sqrt{3n^2-2n+5}-\sqrt{n}}{\sqrt{n^2-n+1}+4n} \right) \\
                        &=&\limp \left( \frac{n}{n} \cdot \frac{\sqrt{3-\frac{2}{n}+\frac{5}{n^2}}-\sqrt{\frac{1}{n^2}}}
                        {\sqrt{1-\frac{1}{n}+\frac{1}{n^2}}+4} \right) \\ \\
                        & \overset{*}{=} & \frac{\sqrt{\limp 3-\frac{2}{n}+\frac{5}{n^2}}-\sqrt{\limp \frac{1}{n^2}}}
                        {\sqrt{\limp 1-\frac{1}{n}+\frac{1}{n^2}}+4} \\
                        &=& \frac{\sqrt{3}}{5}
                    \end{align*}
                    * an dieser Stelle wurde benutzt, dass die Wurzelfunktion stetig ist.
                \item[b)]
                    \begin{align*}
                        &&\limp \left( \cos \left( \frac{\sqrt{10n^2 -n}-n}{2n+3} \right) \right) \\
                        & \overset{*}{=} & \cos \left( \limp \left( \frac{\sqrt{10n^2 -n}-n}{2n+3} \right) \right) \\
                        &=& \cos \left( \limp \left( \frac{n}{n} \cdot \frac{\sqrt{10-\frac{1}{n}}-\frac{n}{n}}{2+\frac{3}{n}} \right) \right) \\
                        & \overset{**}{=} & \cos \left( \frac{\sqrt{\limp 10-\frac{1}{n}}-1}{\limp 2+\frac{3}{n}} \right) \\
                        &=& \cos \left( \frac{\sqrt{10}-1}{2} \right)
                    \end{align*}
                    * an dieser Stelle wurde benutzt, dass die Cosinusfunktion stetig ist. \\
                    ** an dieser Stelle wurde benutzt, dass die Wurzelfunktion stetig ist.
            \end{enumerate}
        % Aufgabe 3
        \item[\textbf{3.}]    \ \\
                Wir wollen zeigen, dass  $g(f(x_n)) \rightarrow g(f(x_0))$  gilt,  somit die Nacheinanderausführung stetiger Funktionen ebenfalls stetig ist.
               $ x_n $ und  $x_0 $ befinden sich für  $x_n \rightarrow x_0 $ beide in der Definitionsmenge von  $g \circ f $.\\
                Da  $f $ an der Stelle  $x_0 $ stetig ist folgt  f(x_n) $\rightarrow f(x_0).$ \\
                Da \ $g$ \  an der Stelle \ $f(x_0)$ \  stetig ist gilt somit:\\ \\
                $$g(f(x_n)) \rightarrow g(f(x_0).$$\\
            

        % Aufgabe 4
        \item[\textbf{4.}]
                $$ \limo (f(x)) = \limo \brac{(\cos\brac{\frac{1}{x}}} = \ cos(\infty) $$ \\
                Geht  $x \rightarrow 0$  wird unser Wert für  $\cos(\frac{1}{x}) \Rightarrow \cos(\infty)$,
                \\somit fängt die Funktion an, immer schneller zu alternieren.\\ (Die Periode von Cosinus bleibt immer konstant). \\ Somit ist die Funktion an der Stelle $x_0 = 0$ nicht stetig.

                $$\limo (g(x)) = \limo \brac{x \cdot \cos\frac{1}{x}} = 0 $$
                Geht $x \rightarrow 0$ nimmt $\brac{(\cos\brac{\frac{1}{x}}}$ auch hier wieder den wert 
                $\cos(\infty)$ an, das verknüpfpte x allerdings nimmt den Wert 0 an, somit geht der ganze Funktionswert ebenfalls gegen 0. \\ Darum ist diese Funktion an der Stelle \ $x = 0$ \ stetig.


	\end{enumerate}


\end{document}

