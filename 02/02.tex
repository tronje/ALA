\newcommand{\authorinfotitle}{Jonathan Siems, 6533519, Gruppe 12\\ Tronje Krabbe, 6435002, Gruppe 9}
\newcommand{\authorinfo}{Jonathan Siems, Tronje Krabbe}
\newcommand{\titleinfo}{ALA 02 17.04.2014}
\newcommand{\qed}{\square}
\newcommand{\todo}{\textbf{\textcolor{red}{TODO}}}

\newcommand{\limp}{\underset{n \to \infty}{\lim}}
\newcommand{\limn}{\underset{n \to - \infty}{\lim}}

\documentclass [a4paper,12pt]{article}
\usepackage[german,ngerman]{babel}
\usepackage[utf8]{inputenc}
\usepackage[T1]{fontenc}
\usepackage{lmodern}
\usepackage{amssymb}
\usepackage{mathtools}
\usepackage{amsmath}
\usepackage{enumerate}
\usepackage{breqn}
\usepackage{fancyhdr}
\usepackage{multicol}
\usepackage{color}

\author{\authorinfotitle}
\title{\titleinfo}
\date{\today}

\pagestyle{fancy}
\fancyhf{}
\fancyhead[R]{\authorinfo}
\fancyhead[L]{ALA Hausaufgaben}
\fancyfoot[C]{\thepage}
\begin{document}
\maketitle
    \begin{enumerate}
        % Aufgabe 1
        \item[\textbf{1.}]
            \begin{enumerate}
                \item[(i)]
                        $$\limp
                        \left( \frac{-3n^4+2n^2+n+1}{-7n^4+25} \right) \\
                        \Leftrightarrow \ \ \limp
                        \left( \frac{n^4}{n^4} \cdot \frac{-3+ \frac{2}{n^2} + \frac{1}{n^3} + \frac{1}{n^4}}{-7+ \frac{25}{n^4}} \right) \\
                        \Leftrightarrow \ \ \frac{3}{7}$$
                \item[(ii)]
                    
                        $$\limp
                        \left( \frac{-3n^4+2n^2+n+1}{-7n^5+25} \right) \\
                        \Leftrightarrow \  \ \limp
                        \left( \frac{1}{n^5} \cdot \frac{-3+ \frac{2}{n^2} + \frac{1}{n^3} + \frac{1}{n^4}}{-7+ \frac{25}{n^5}} \right) \\
                        \Leftrightarrow \ \ 0$$
                \item[(iii)]
                        $$\limp
                        \left( \frac{-3n^5+2n^2+n+1}{-7n^4+25} \right) \\
                        \Leftrightarrow \ \  \limp
                        \left( \frac{n}{1} \cdot \frac{-3+ \frac{2}{n^3} + \frac{1}{n^4} + \frac{1}{n^5}}{-7+ \frac{25}{n^4}} \right) \\
                        \Leftrightarrow \ \ \infty$$
                   
                \item[(iv)]
                    \begin{align*}
                        &\limp
                        \left( \frac{6n^3+2n-3}{9n^2+2} - \frac{2n^3+5n^2+7}{3n^2+3} \right) \\[0,5cm]
                        \Leftrightarrow \ &\limp
            			%\left( \frac{-18n^5-45n^4-63n^2-4n^3-10n^2-14+18n^5+6n^3-9n^2+18n^3+6n-9}{(9n^2+2)\cdot(3n^2+3)} \right) \\
            			%\Leftrightarrow \ &\limp
                        %ich wuerde diese ewig lange Fraction einfach auslassen.
            			\left( \frac{-45n^4+20n^3-82n^2+6n-23}{27n^4+33n^2+6} \right) \\[0,5cm]
            			\Leftrightarrow \ &\limp
            			\left( \frac{n^4}{n^4}\cdot\frac{-45+\frac{20}{n}-\frac{82}{n^2}+\frac{6}{n^3}-\frac{23}{n^4}}{27+\frac{33}{n^2}+\frac{6}{n^4}} \right) 
            			\Leftrightarrow \ &-\frac{45}{27} \ \Leftrightarrow \ -\frac{5}{3} \\
                    \end{align*}
		        \item[(v)]
                    \begin{align*}
            			&\limp \left(\frac{\sqrt{9n^4+n^2+1}-2n^2+3}{\sqrt{2n^2+1} \cdot \sqrt{2n^2+n+1}} \right) \\
            			\Leftrightarrow \ &\limp 
            			\left( \frac{\sqrt{9n^4+n^2+1}-2n^2+3}{\sqrt{4n^4+2n^3+4n^2++n+1}} \right) \\
            			\Leftrightarrow \ &\limp
            			\left( \frac{n^4}{n^4} \cdot \frac{\sqrt{9+\frac{1}{n^2}+\frac{1}{n^4}}-\frac{2}{n^2}+\frac{3}{n^4}}{\sqrt{4+\frac{2}{n}+\frac{4}{n^2}+\frac{1}{n^3}+\frac{1}{n^4}}}\right) 
            			 \Leftrightarrow  &\frac{\sqrt{9}}{\sqrt{4}}  \Leftrightarrow \ \frac{3}{2} %Not sure if correct, or bullshit.
                    \end{align*}
            \end{enumerate}
        % Aufgabe 2
        \item[\textbf{2.}]
            \begin{enumerate}
                \item[a)]
                    \begin{enumerate}
                        \item[(i)]
                            \begin{align*}
                                &a_0 \ = \ 1		 	&&  	s_0 \ = \ 1\\[0,15cm]
                                &a_1 \ = \ \frac{2}{5} 		&&	s_1 = \frac{7}{5}\\[0,15cm]
                                &a_2 \ = \ \frac{4}{25} 	&&	s_2 = \frac{39}{25}\\[0,15cm]
                                &a_3 \ = \ \frac{8}{125} 	&&	s_3 = \frac{203}{125}\\[0,15cm]
                                &a_4 \ = \ \frac{16}{625}	&&	s_4 = \frac{1031}{625} 
                            \end{align*}
				 $$\text{Diese geometrische Reihe konvergiert, da q =} \ \frac{2}{5} \Rightarrow |q| \ < 1$$
				\[\text{Sie konvergiert gegen} \ \  \sum_{i=0}^\infty~\left(\frac{2}{5}\right) = \frac{1}{1-\frac{2}{5}} = \frac{5}{3} \]
                        \item[(ii)]
                            \begin{align*}
				&a_0 \ = \ 1				&&	s_0 = 1\\[0,15cm]
				&a_1 \ = \ \frac{5}{2}		&&	s_1 = \frac{7}{2}\\[0,15cm]
				&a_2 \ = \ \frac{25}{4}		&&	s_2 = \frac{39}{4}\\[0,15cm]
				&a_3 \ = \ \frac{125}{8}	&&	s_3 = \frac{203}{8}\\[0,15cm]
				&a_4 \ = \ \frac{625}{16}	&&	s_4 = \frac{1031}{16} \\[0,15cm]
                            \end{align*}
				$$\text{Diese geometrische Reihe divergiert, da q =} \ \frac{5}{2} \Rightarrow |q| \ > 1$$
                        \item[(iii)]
                            \begin{align*}
		                &a_0 \ = \ \  1			&&      s_0 = 1\\[0,15cm]
		                &a_1 \ =  -\frac{2}{5}		&&	s_1 = \frac{3}{5}\\[0,15cm]
		                &a_2 \ = \ \  \frac{4}{25} 	&&     	s_2 = \frac{19}{25}\\[0,15cm]
		                &a_3 \ =  -\frac{8}{125}	&&      s_3 = \frac{87}{125}\\[0,15cm]
		                &a_4 \ = \ \  \frac{16}{625}	&&      s_4 = \frac{451}{625} \\
                            \end{align*}
				$$\text{Diese geometrische Reihe konvergiert, da q =} \ -\frac{2}{5} \Rightarrow |q| \ < 1 $$ 
				\[\text{Sie konvergiert gegen} \ \  \sum_{i=0}^\infty~\left(-\frac{2}{5}\right) = \frac{1}{1-(-\frac{2}{5})} = \frac{5}{7} \]
                    \end{enumerate}
                \item[b)]
			\begin{enumerate}
			    \item[(i)]
							 \[ \sum_{i=0}^\infty~x^i = \frac{1}{1-(-\frac{3}{10})} = \frac{10}{13} \]
			    \item[(ii)]
			    	\begin{align*}
						 \sum_{i=0}^\infty~x^i \ = \ \frac{1}{1-x} \ = \ \frac{5}{8} \quad &\Leftrightarrow \quad 1  = \ (1-x) \ \cdot \ \frac{5}{8} \\[0,5cm]
						 \Leftrightarrow   \  1  = \frac{5}{8} \ - \ \frac{5}{8}x \quad &\Leftrightarrow \quad \frac{3}{8}  = \ -\frac{5}{8}x \quad \Leftrightarrow \quad x = -\frac{3}{5}
					\end {align*}	
			\end{enumerate}
            \end{enumerate}
        % Aufgabe 3
        \item[\textbf{3.}]
        		\item[(i)]
        			$$\sum_{i=0}^\infty~\left(\frac{5}{8} \right)^i \text{konvergiert, da q =} \frac{5}{8} \Rightarrow |q| \ < \ 1$$ 
        			$$\sum_{i=0}^\infty~\left(\frac{5}{8} \right)^i \ = \ \frac{1}{1 -\frac{5}{8}} \ = \ \frac{3}{8} $$
        		\item[(ii)]
        			$$\sum_{i=2}^\infty~\left(\frac{5}{8} \right)^i \todo$$ 
        		\item[(iii)]
        			$$\sum_{i=1}^\infty~\left(-\frac{5}{8} \right)^i \todo$$
        		\item[(iv)]
            		$$\sum_{i=1}^\infty~(-1)^i  \cdot  \left( \frac{5}{8} \right)^{i+2} \todo$$
        % Aufgabe 4
        \item[\textbf{4.}]
        		\item[(i)]
        			
        			$$\limp \left(1+\frac{1}{n} \right)^5 \Rightarrow (1+0)^5 = 1 \cdot 1 \cdot 1 \cdot 1 \cdot 1 \ = \ 1 $$  \\
        		\item[(ii)]
        			$$\limp \left( 1 + \frac{1}{n} \right)^{n+5} \Rightarrow \limp \left(1+\frac{1}{n} \right)^n \cdot \limp \left(1+\frac{1}{n} \right)^5 \ \Rightarrow \  e \cdot 1^5 \ = \ e  	$$ \\
        		\item[(iii)] 
        		$$ \limp \left( 1 + \frac{1}{n} \right)^{2n+3} \Rightarrow \left(\limp \left( 1 + \frac{1}{n} \right)^n \right)^2 \cdot \left(1+ \frac{1}{n} \right)^3 \ = \ e^2 \cdot 1^3 \ = \ e^2 $$
        		\item[(iv)]    
        		$$ \sum_{i=3}^\infty~\frac{1}{i(i+1)} \ \Leftrightarrow \ \sum_{1=3}^\infty~\frac{1}{i}\cdot\frac{1}{i+1} \Leftrightarrow \sum_{1=3}^\infty~\frac{1}{i} \cdot \sum_{1=4}^\infty~\frac{1}{i} \ = \ \infty \cdot \infty \ = \ \infty $$ 			
            
    \end{enumerate}

\end{document}
