\newcommand{\authorinfo}{Jonathan Siems, Lina, Tronje Krabbe}
\newcommand{\titleinfo}{ALA 01 (HA) zum 10.04.2014}
\newcommand{\qed}{\square}

\documentclass [a4paper,11pt]{article}
\usepackage[german,ngerman]{babel}
\usepackage[utf8]{inputenc}
\usepackage[T1]{fontenc}
\usepackage{lmodern}
\usepackage{amssymb}
\usepackage{mathtools}
\usepackage{amsmath}
\usepackage{enumerate}
\usepackage{breqn}
\usepackage{fancyhdr}
\usepackage{multicol}

\author{\authorinfo}
\title{\titleinfo}
\date{\today}

\pagestyle{fancy}
\fancyhf{}
\fancyhead[L]{\authorinfo}
\fancyhead[R]{\titleinfo}
\fancyfoot[C]{\thepage}

\begin{document}
\maketitle
    \begin{enumerate}
        % Aufgabe 1
        \item[\textbf{1.}]
            Wir unterscheiden zwei Fälle: \\
            Fall $x > -5$:\\
            \begin{align}
                \frac{2}{x+5} \geq 3 \\
                \Leftrightarrow 2 \geq 3x+15 \\
                \Leftrightarrow x \leq (- \frac{13}{3})
            \end{align}
            Fall $x < -5$:\\
                In diesem Fall ist die gesamte Linke Seite der Ungleichung immer negativ, da der Nenner des Bruches negativ ist. Das bedeutet, dass für x kleiner -5 keine Lösung existiert.
            % Quatsch:
            % \begin{align}
            %     \frac{2}{x+5} \geq 3 \\
            %     \Leftrightarrow 2 \leq 3x+15 \\
            %     \Leftrightarrow x \geq (- \frac{13}{3})
            % \end{align}
            Daraus folgt: \\ \\
            $L = (- 5, - \frac{13}{3}]$
        % Aufgabe 2
        \item[\textbf{2.}]
            Wir unterscheiden abermals zwei Fälle: \\
            Fall $x \geq \frac{4}{3}$: \\
            \begin{align}
                |3x-4| \geq 2 \\
                \Leftrightarrow x \geq 2
            \end{align}
            Fall $x < \frac{4}{3}$: \\
            \begin{align}
                |3x-4| \geq 2 \\
                \Leftrightarrow -(3x-4) \geq 2 \\
                \Leftrightarrow x \leq \frac{2}{3}
            \end{align}
            Daraus folgt: \\ \\
            $L = (- \infty, \frac{2}{3}] \cup (2, \infty)$
        % Aufgabe 3
        \item[\textbf{3.}]
            \begin{enumerate}
                \item[a)]
                    \begin{align}
                        |a_n - a| &=& |\frac{4n-1}{n+5} - 4| \\
                        &=& |\frac{4n-1}{n+5} - \frac{4(n+5)}{n+5}| \\
                        &=& |\frac{4n-1}{n+5} - \frac{4n+20}{n+5}| \\
                        &=& |\frac{-21}{n+5}| \\
                        &=& - \frac{21}{n+5}
                    \end{align}
                \item[b)]
                    Gesucht ist:
                    \begin{align}
                        |a_n - a| < \epsilon \Leftrightarrow - \frac{21}{n+5} < \epsilon \\
                        \Leftrightarrow n > - \frac{21}{\epsilon} -5
                    \end{align}
                    Nun setze man $N > - \frac{21}{\epsilon} -5$. Daraus folgt: \\
                    für $n \geq N$ gilt $|a_n - a| < \epsilon$. \\
                    Somit ist gezeigt, dass die Folge $a_n$ gegen $a$ konvergiert.
                \item[c)]
                    \begin{align}
                        & \epsilon = \frac{1}{10} \Rightarrow N = -214 \\
                        & \epsilon = \frac{1}{100} \Rightarrow N = -2104 \\
                        & \epsilon = \frac{1}{1000} \Rightarrow N = -21004 \\
                    \end{align}
            \end{enumerate}
        % Aufgabe 4
        \item[\textbf{4.}]
            \textbf{Beschränktheit}\\
                (1) Behauptung: $1 \leq a_n < 2$ \\
                (1) gilt für $n=1$: $1 \leq a_1= \frac{5}{3} < 2$ \\
                Angenommen, (1) gilt für ein beliebiges, fest gewähltes $n$. Dann muss auch gelten:\\
                \begin{align}
                    1 \leq a_{n+1} < 2 \\
                    \Leftrightarrow 1 \leq ( \frac{a_n}{2} )^2 + 1 < 2
                \end{align}
                Zeige zuerst, dass gilt: $1 \leq a_{n+1}$
                \begin{align}
                    1 \leq ( \frac{a_n}{2} )^2 + 1 \\
                    \Leftrightarrow 0 \leq ( \frac{a_n}{2} )^2 \\
                    \Leftrightarrow 0 \leq \frac{a_n}{2} \\
                    \Leftrightarrow 0 \leq a_n
                \end{align}
                Dies entspricht unserer Annahme (1).\\
                Zeige nun, dass gilt: $a_{n+1} < 2$
                \begin{align}
                    ( \frac{a_n}{2} )^2 + 1 < 2 \\
                    \Leftrightarrow ( \frac{a_n}{2} )^2 < 1 \\
                    \Leftrightarrow \frac{a_n}{2} < 1 \\
                    \Leftrightarrow a_n < 2
                \end{align}
                Dies fügt sich ebenfalls (1). Somit ist durch vollständige Induktion gezeigt, dass $(a_n)$ beschränkt ist. \\
                \\
            \textbf{Monotonie}\\
                Zu zeigen: $a_{n+1} \geq a_n$ für alle $n \in \mathbb{N}$. (2)\\
                Wir verwenden abermals vollständige Induktion um diese Aufgabe zu lösen: \\
                (2) gilt für $n=1$: $\frac{61}{36} \geq \frac{5}{3}$. \\
                Angenommen, (2) gilt für ein beliebiges, fest gewähltes $n$. Dann muss auch gelten:
                \begin{align}
                    a_{n+2} &\geq a_{n+1} \\
                    \Leftrightarrow ( \frac{a_{n+1}}{2} )^2 + 1 &\geq ( \frac{a_n}{2} )^2 + 1 \\
                    \Leftrightarrow \frac{a_{n+1}}{2} &\geq \frac{a_n}{2} \\
                    \Leftrightarrow a_{n+1} &\geq a_n
                \end{align}
                Hiermit ist nun auch die Monotonie der Folge bewiesen.
            
            

    \end{enumerate}


\end{document}