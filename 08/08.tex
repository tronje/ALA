\newcommand{\authorinfotitle}{Jonathan Siems, 6533519, Gruppe 12\\ Jan-Thomas Riemenschneider, 6524390, Gruppe 12 \\ Tronje Krabbe, 6435002, Gruppe 9}
\newcommand{\authorinfo}{Jonathan Siems, Tronje Krabbe, Jan-Thomas Riemenschneider}

\newcommand{\qed}{\square}
\newcommand{\limp}{\underset{n \to \infty}{\lim}}
\newcommand{\limn}{\underset{n \to - \infty}{\lim}}
\newcommand{\todo}{\textbf{\textcolor{red}{TODO}}}
\newcommand{\todaynum}{\the\day.\the\month.\the\year}
\newcommand{\abgabe}{\textbf{\textcolor{red}{ABGABEDATUM}}}
\newcommand{\titleinfo}{ALA BLATTNR. 08 19.06.2014}

\documentclass[a4paper,11pt]{article}
\usepackage[a4paper]{geometry}
\usepackage[german,ngerman]{babel}
\usepackage[utf8]{inputenc}
\usepackage[T1]{fontenc}
\usepackage{amsmath,amssymb,amstext}
\usepackage{mathtools}
\usepackage{enumerate}
\usepackage{breqn}
\usepackage{fancyhdr}
\usepackage{multicol}
\usepackage{color}
\usepackage{microtype}
\usepackage{booktabs}
\usepackage{lmodern}
\usepackage{tikz}
\usepackage{pgfplots}
\usepackage{scrdate}
\usepackage[makeroom]{cancel}
\usetikzlibrary{calc}

\title{\titleinfo}
\author{\authorinfotitle}
\date{\today}

\pagestyle{fancy}
\fancyhf{}
\fancyhead[R]{\authorinfo}
\fancyhead[L]{ALA Hausaufgaben}
\fancyfoot[C]{\thepage}


\begin{document}
\maketitle
    \begin{enumerate}
        % Aufgabe 1
        \item[\textbf{1.}]
            \begin{enumerate}
                \item[a)]
                    \begin{align*}
                        &T_7(x) = 1-\frac{x^2}{2}+\frac{x^4}{24}-\frac{x^6}{720}\\
                        &T_8(x) = 1-\frac{x^2}{2}+\frac{x^4}{24}-\frac{x^6}{720}+\frac{x^8}{8!}\\
                        &T_9(x) = T_8(x)\\
                        &T_{10}(x) = 1-\frac{x^2}{2}+\frac{x^4}{24}-\frac{x^6}{720}+\frac{x^8}{8!}-\frac{x^10}{10!}\\
                        &T_{11}(x) = T_{10}(x)\\
                        &T_{12}(x) = 1-\frac{x^2}{2}+\frac{x^4}{24}-\frac{x^6}{720}+\frac{x^8}{8!}-\frac{x^10}{10!}+\frac{x^12}{12!}\\
                        &T_{13}(x) = T_{12}(x)\\
                        \\
                        &T_9(1) \approx 0,5403025\\
                        &T_{11}(1) \approx 0,5403023\\
                        &T_{13}(1) \approx 0,5403023
                    \end{align*}
                \item[b)]
                    \begin{align*}
                        &\textbf{f(x)}\\
                        &T_0(x)= 1\\
                        &T_1(x)= 1+ \frac{x}{2}\\
                        &T_2(x)= 1+ \frac{x}{2}- \frac{x^2}{8}\\
                        &T_3(x)= 1+ \frac{x}{2}- \frac{x^2}{8}+ \frac{x^3}{16}\\
                        &T_4(x)= 1+ \frac{x}{2}- \frac{x^2}{8}+ \frac{x^3}{16} - \frac{5x^4}{128}\\
                        \\
                        &\textbf{g(x)}\\
                        &T_0(x)= 1\\
                        &T_1(x)= 1- \frac{x}{3}\\
                        &T_2(x)= 1- \frac{x}{3}+ \frac{2x^2}{9}\\
                        &T_3(x)= 1- \frac{x}{3}+ \frac{2x^2}{9} - \frac{14x^3}{81}\\
                        &T_4(x)= 1- \frac{x}{3}+ \frac{2x^2}{9} - \frac{14x^3}{81} + \frac{35x^4}{243}\\
                    \end{align*}
                \item[c)]
                    Am einfachsten ist es, einfach alle Taylorpolynome bis $T_5$ zu errechnen, da diese Arbeite sowieso
                    getan werden muss.
                    \begin{align*}
                        &T_0(x) = 0\\
                        &T_1(x) = x\\
                        &T_2(x) = x+x^2\\
                        &T_3(x) = x+x^2+ \frac{1}{3}x^3\\
                        &T_4(x) = x+x^2+ \frac{1}{3}x^3\\
                        &T_5(x) = x+x^2+ \frac{1}{3}x^3 - \frac{1}{30}x^5\\
                        &\text{Probe:}\\
                        &T_5(1) = \frac{69}{30}=2,3\\
                        &f(1) = 2,287355...
                    \end{align*}
                    Das Ergebnis kommt also hin.
            \end{enumerate}
              
        % Aufgabe 2
        \item[\textbf{2.}]
            \begin{enumerate}
                \item[(i)]
                    \begin{align*}
                        &\underset{x \to 1}{\lim} \left( \frac{x^3-3x^2+x+2}{x^2-5x+6} \right) = \frac{1}{2}
                    \end{align*}
                \item[(ii)]
                    \begin{align*}
                        &\underset{x \to 2}{\lim} \left( \frac{x^3-3x^2+x+2}{x^2-5x+6} \right)\\
                        \overset{*}{=}&\underset{x \to 2}{\lim} \left( \frac{3x^2-6x+1}{2x-5} \right)\\
                        =& -1
                    \end{align*}
                    * An dieser Stelle wurden die Regeln von de l'Hospital verwendet.
                \item[(iii)]
                    \begin{align*}
                        &\underset{x \to 0}{\lim} (1+3x)^{\frac{1}{2x}}\\
                        =&\underset{x \to 0}{\lim} \left( e^{\frac{1}{2x} \cdot \ln(1+3x)} \right)\\
                        =&e^{\underset{x \to 0}{\lim} \left( \frac{1}{2x} \cdot \ln(1+3x) \right)}
                    \end{align*}
                    Wir errechnen zunächst nur die Potenz:
                    \begin{align*}
                        &\underset{x \to 0}{\lim} \left( \frac{\ln(1+3x)}{2x} \right)\\
                        \overset{*}{=}&\underset{x \to 0}{\lim} \left( \frac{\frac{3}{3x+1}}{2} \right)\\
                        =& \frac{3}{2}\\
                    \end{align*}
                    Wir setzen dieses Zwischenergebnis ein und erhalten das Endergebnis:\\
                        $\Rightarrow e^{\frac{3}{2}}$\\
                    * An dieser Stelle wurden die Regeln von de l'Hospital verwendet.
                \item[(iv)]
                    \begin{align*}
                        &\underset{x \to 0}{\lim} \left( \frac{1}{e^x -1}-\frac{1}{\sin(x)} \right)\\
                        =&\underset{x \to 0}{\lim} \left( \frac{\sin(x) - e^x + 1}{e^x \sin(x) - \sin(x)} \right)\\
                        \overset{*}{=}&\underset{x \to 0}{\lim} \left( \frac{\cos(x) - e^x}{e^x \sin(x) +(e^x -1) \cos(x)} \right)\\
                        \overset{*}{=}&\underset{x \to 0}{\lim} \left( \frac{-e^x - \sin(x)}{\sin(x) +2e^x \cos(x)} \right)\\
                        =& -\frac{1}{2}
                    \end{align*}
                    * An dieser Stelle wurden die Regeln von de l'Hospital verwendet.
            \end{enumerate}

        % Aufgabe 3
        \item[\textbf{3.}]
            \begin{enumerate}
                \item[a)]
                    Die Steigung von $t$ ist die Steigung von $f$ an der Stelle $(2,9)$.
                    \begin{align*}
                        f(x)=x^3-x^2+3x-1\\
                        f'(x)=3x^2-2x+3\\
                        f'(2)=11
                    \end{align*}
                    Die Steigung von $t$ ist also 11. Damit wissen wir $t(2)=9$, sowie z.B. $t(3)=20$. Also:
                    \begin{align*}
                        t(x)=ax+b\\
                        9=2a+b\\
                        20=3a+b\\
                        b=9-2a\\
                        20=3a+9-2a\\
                        20=a+9\\
                        a=11\\
                        b=-13\\
                        t(x)=11x-13\\
                        t(x)=0\\
                        \Leftrightarrow x=\frac{13}{11}
                    \end{align*}
                    $t$ schneidet also die $x$-Achse an der Stelle $\frac{13}{11}$
                \item[b)]
                    \begin{align*}
                        &h(x)=x^x\\
                        &h'(x)=x^x (\ln(x)+1)\\
                        &h''(x)=x^x \left( \frac{1}{x} (\ln(x)+1)^2 \right)\\
                        &h'(0)=- \infty
                    \end{align*}
                    Well shit
                \item[c)]
                    Im Folgenden bilden wir die ersten drei Ableitungen von $\sqrt[5]{x+1}$ und berechnen die Funktionswerte für $x = 0$,
                    die anschliessend in die Formel für Taylorpolynome eingesetzt werden:
                    \begin{align*}
                    &f(x) = \sqrt[5]{x+1} && f(0) = 1 \\
                    &f'(x) = \frac{1}{5}(x+1)^{-\frac{4}{5}} && f'(0) = \frac{1}{5} \\
                    &f''(x) = -\frac{4}{25}(x+1)^{-\frac{9}{5}} && f''(0) = -\frac{4}{25} \\
                    &f'''(x) = \frac{36}{125}(x+1)^{-\frac{14}{5}} && f'''(0) = \frac{36}{125} \\
                    \end{align*}
                    Einsetzen in $$\sum_{k=0}^{n} \frac{f^{(k)}(0)}{k!}x^k:$$ 
                    \begin{align*}
                    &T_0(x) = 1 \\
                    &T_1(x) = 1 + \frac{1}{5}x \\
                    &T_2(x) = 1 + \frac{1}{5}x -\frac{2}{25}x^2 \\
                    &T_3(x) = 1 + \frac{1}{5}x -\frac{2}{25}x^2 + \frac{6}{125}x^3 \\
                    \end{align*}
                \item[d)]
                    Die Funktion $h$ ist genau dann differenzierbar, wenn der Grenzwert des Differenzenquotienten existiert (mit $x_0=0$):
                    \begin{align*}
                        &\underset{x \to 0}{\lim} \frac{h(x)-h(0)}{x}\\
                        =\ &\underset{x \to 0}{\lim} \frac{x \cdot \cos(\frac{1}{x})}{x}\\
                        \overset{*}{=}\ &\underset{x \to 0}{\lim} \frac{\sin(\frac{1}{x})}{x} + \cos{\frac{1}{x}}
                    \end{align*}
                    *Anwendung von de l'Hospital\\
                    Demnach ist die Funktion an der Stelle 0 nicht differenzierbar, da der Grenzwert zwischen $-\infty$ und $\infty$ schwankt.
            \end{enumerate}

        % Aufgabe 4
        \item[\textbf{4.}]
            \begin{enumerate}
                \item[a)]
                    \begin{align*}
                        &\underset{x \to \infty}{\lim} \left( \frac{a^x}{x^n} \right)\\
                        =\ &\underset{x \to \infty}{\lim} \left( \frac{xa^{x-1}}{nx^{n-1}} \right)\\
                        % =\ &\underset{x \to \infty}{\lim} \left( \frac{(x^2-x)a^{x-2}}{(n^2-n)x^{n-2}} \right)\\
                    \end{align*}
                    Nach de l'Hospital darf man auch die Ableitungen beider Funktionen vergleichen. Dies kann $n$-Mal fortgeführt
                    werden, bis:
                    \begin{align*}
                        &\underset{x \to \infty}{\lim} \left( \frac{p(x) \cdot a^{x-n}}{q(n) \cdot x^0} \right)
                    \end{align*}
                    $p(x)$  ist ein Polynom $n$-ten Grades, wobei das Gleid mit Grad $n$ positiv ist (es ``beginnt'' also mit $x^n$).
                    $q(n)$ ist ein Polynom $n$-ten Grades, ebenfalls positiv. Nun ist bereits eindeutig gezeigt, dass $f$ schneller wächst
                    als $g$, da der Zähler des obigen Bruches der Form $x^n+...$ ist, was, mit $x \to \infty$, viel größer, nämlich unendlich, ist,
                    als der Nenner, welcher lediglich der Form $n^n+...$ ist. $\qed$
                \item[b)]
                    Wir betrachten:
                    \begin{align*}
                        &\underset{x \to \infty}{\lim} \left( \frac{x^r}{\ln^k x} \right)\\
                        =\ &\underset{x \to \infty}{\lim} \left( \frac{rx^{r-1}}{\frac{k \cdot \ln^{k-1}x}{x}} \right)\\
                        =\ &\underset{x \to \infty}{\lim} \left( \frac{rx^{r}}{k \cdot \ln^{k-1}x} \right)
                    \end{align*}
                    Der ``Trick'', der hier angewandt wurde, ist, dass sich nach jedem durch de l'Hospital erlaubten Ableiten
                    beider Funktionen, im Nenner von $h'$ bzw $h'', h'''$ usw. ein $x$ befindet. Dies kann dann in den Z"ahler ``hochgeschoben''
                    werden. Nach der maximalen Anzahl Ableitungen ist der Z"ahler also bedeutend gr"oßer als der Nenner.
                    $g$ wächst also schneller als $h$.
                \item[c)]
                    \begin{enumerate}
                        \item[(i)]
                            Die in a) angewandte Methode funktioniert auch mit $g(x)=x^r$. Sollte $r$ nicht in $\mathbb{N}$ liegen, so wird einfach
                            $\lceil r \rceil$ Male abgeleitet.
                        \item[(ii)]
                            
                    \end{enumerate}
            \end{enumerate}

        % Aufgabe 5
        \item[\textbf{5.}]
            Wir berechnen im Folgenden die Näherungswerte für die Fälle \\ n = 4, n = 5, n = 10 : \\[0.5cm]
            \underline{n = 4}
            $$ \int\limits_{0}^{1} sin \ x \ dx \approx \frac{1}{8}\left(sin(0) + 2 \ sin \left(\frac{1}{4}\right)+ 2 \ sin\left(\frac{2}{4}\right) + 2 \ sin\left(\frac{3}{4}\right)+ sin \ (1)\right)$$
            $$ \approx 0.45730093.$$ 

             \underline{n = 5}
            $$ \int\limits_{0}^{1} sin \ x \ dx \approx \frac{1}{10}\left(sin(0) +  2 sin\left(\frac{1}{5}\right) + 2sin\left(\frac{2}{5}\right) + 2 sin\left( \frac{3}{5}\right) + 2 sin\left( \frac{4}{5}\right) + sin(1)\right) $$
            $$ \approx 0.45816434 $$

            \underline{n = 10}
            \begin{align*}
            \int\limits_{0}^{1} sin \ x \ dx \approx &\frac{1}{20}\Bigg(sin(0) + 2 \ sin \left(\frac{1}{10}\right)+ 2 \ sin \left(\frac{2}{10}\right)+ 2 \ sin \left(\frac{3}{10}\right)+ 2 \ sin \left(\frac{4}{10}\right) \\ 
            &+ 2 \ sin \left(\frac{5}{10}\right)+ 2 \ sin \left(\frac{6}{10}\right)+ 2 \ sin \left(\frac{7}{10}\right)+ 2 \ sin \left(\frac{8}{10}\right)+ 2 \ sin \left(\frac{9}{10}\right)+  sin(0)\Bigg) \\[0.5cm]
            &\approx 0.4593145488579763249099
            \end{align*}
        % Aufgabe 6
        \item[\textbf{6.}]
            \begin{enumerate}
                \item [d)]
                \begin{align*}
                    f(x) &= \frac{1}{x^2+1} \\[0.5cm]
                    f'(x) &= \frac{0 \cdot (x^2+1) - 1 \cdot 2x}{(x^2+1)^2} \\
                    &= \frac{-2x}{(x^2+1)^2} \\[0.5cm]
                    f''(x) &= \frac{-2 \cdot (x^2 +1)^2 - (-2x)4x(x^2+1)}{(x^2+1)^4} \\
                    &= \frac{-2(x^2+1)^2 - (-8x^2)(x^2+1)}{(x^2+1)^4} \\
                    &= \frac{\cancel{(x^2+1)}(-2(x^2+1)-(-8x^2))}{(x^2+1)^3} \\
                    &= \frac{-2x^2-2+8x^2}{(x^2+1)^3} \\
                    &= \frac{6x^2-2}{(x^2+1)^3} \\[0.5cm]
                    f'''(x) &= \frac{12x(x^2+1)^3 -(6x^2-2)6x(x^2+1)^2}{(x^2+1)^6} \\
                    &= \frac{\cancel{(x^2+1)^2}(12x(x^2+1)-(36x^3-12x))}{(x^2+1)^4}
                    &= \frac{12x^3+12x-(36x^3-12x)}{(x^2+1)^4} \\
                    &= \frac{-24x^3+24x}{(x^2+1)^4} \\
                \end{align*}
            \end{enumerate}

    \end{enumerate}


\end{document}

