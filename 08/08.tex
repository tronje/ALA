\newcommand{\authorinfotitle}{Jonathan Siems, 6533519, Gruppe 12\\ Jan-Thomas Riemenschneider, 6524390, Gruppe 12 \\ Tronje Krabbe, 6435002, Gruppe 9}
\newcommand{\authorinfo}{Jonathan Siems, Tronje Krabbe, Jan-Thomas Riemenschneider}

\newcommand{\qed}{\square}
\newcommand{\limp}{\underset{n \to \infty}{\lim}}
\newcommand{\limn}{\underset{n \to - \infty}{\lim}}
\newcommand{\todo}{\textbf{\textcolor{red}{TODO}}}
\newcommand{\todaynum}{\the\day.\the\month.\the\year}
\newcommand{\abgabe}{\textbf{\textcolor{red}{ABGABEDATUM}}}
\newcommand{\titleinfo}{ALA BLATTNR. \abgabe}

\documentclass[a4paper,11pt]{article}
\usepackage[a4paper]{geometry}
\usepackage[german,ngerman]{babel}
\usepackage[utf8]{inputenc}
\usepackage[T1]{fontenc}
\usepackage{amsmath,amssymb,amstext}
\usepackage{mathtools}
\usepackage{enumerate}
\usepackage{breqn}
\usepackage{fancyhdr}
\usepackage{multicol}
\usepackage{color}
\usepackage{microtype}
\usepackage{booktabs}
\usepackage{lmodern}
\usepackage{tikz}
\usepackage{pgfplots}
\usepackage{scrdate}
\usetikzlibrary{calc}

\title{\titleinfo}
\author{\authorinfotitle}
\date{\today}

\pagestyle{fancy}
\fancyhf{}
\fancyhead[R]{\authorinfo}
\fancyhead[L]{ALA Hausaufgaben}
\fancyfoot[C]{\thepage}


\begin{document}
\maketitle
    \begin{enumerate}
        % Aufgabe 1
        \item[\textbf{1.}]
            \begin{enumerate}
                \item[a)]
                    \begin{align*}
                        &T_7(x) = 1-\frac{x^2}{2}+\frac{x^4}{24}-\frac{x^6}{720}\\
                        &T_8(x) = 1-\frac{x^2}{2}+\frac{x^4}{24}-\frac{x^6}{720}+\frac{x^8}{8!}\\
                        &T_9(x) = T_8(x)\\
                        &T_{10}(x) = 1-\frac{x^2}{2}+\frac{x^4}{24}-\frac{x^6}{720}+\frac{x^8}{8!}-\frac{x^10}{10!}\\
                        &T_{11}(x) = T_{10}(x)\\
                        &T_{12}(x) = 1-\frac{x^2}{2}+\frac{x^4}{24}-\frac{x^6}{720}+\frac{x^8}{8!}-\frac{x^10}{10!}+\frac{x^12}{12!}\\
                        &T_{13}(x) = T_{12}(x)\\
                        \\
                        &T_9(1) \approx 0,5403025\\
                        &T_{11}(1) \approx 0,5403023\\
                        &T_{13}(1) \approx 0,5403023
                    \end{align*}
                \item[b)]
                    \begin{align*}
                        &\textbf{f(x)}\\
                        &T_0(x)= 1\\
                        &T_1(x)= 1+ \frac{x}{2}\\
                        &T_2(x)= 1+ \frac{x}{2}- \frac{x^2}{8}\\
                        &T_3(x)= 1+ \frac{x}{2}- \frac{x^2}{8}+ \frac{x^3}{16}\\
                        &T_4(x)= 1+ \frac{x}{2}- \frac{x^2}{8}+ \frac{x^3}{16} - \frac{5x^4}{128}\\
                        \\
                        &\textbf{g(x)}\\
                        &T_0(x)= 1\\
                        &T_1(x)= 1- \frac{x}{3}\\
                        &T_2(x)= 1- \frac{x}{3}+ \frac{2x^2}{9}\\
                        &T_3(x)= 1- \frac{x}{3}+ \frac{2x^2}{9} - \frac{14x^3}{81}\\
                        &T_4(x)= 1- \frac{x}{3}+ \frac{2x^2}{9} - \frac{14x^3}{81} + \frac{35x^4}{243}\\
                    \end{align*}
                \item[c)]
                    Am einfachsten ist es, einfach alle Taylorpolynome bis $T_5$ zu errechnen, da diese Arbeite sowieso
                    getan werden muss.
                    \begin{align*}
                        T_0(x) = 0\\
                        T_1(x) = x\\
                        T_2(x) = x+x^2\\
                        T_3(x) = x+x^2+ \fac{1}{3}x^3\\
                        T_4(x) = x+x^2+ \fac{1}{3}x^3\\
                        T_5(x) = x+x^2+ \fac{1}{3}x^3 - \frac{1}{30}x^5\\
                        \text{Probe:}\\
                        T_5(1) = \frac{69}{30}=2,3\\
                        f(1) = 2,287355...
                    \end{align*}
                    Das Ergebnis kommt also hin.
            \end{enumerate}
              
        % Aufgabe 2
        \item[\textbf{2.}]
            \todo

        % Aufgabe 3
        \item[\textbf{3.}]
            \todo

        % Aufgabe 4
        \item[\textbf{4.}]
            \todo

        % Aufgabe 5
        \item[\textbf{5.}]
            \todo

        % Aufgabe 6
        \item[\textbf{6.}]
            \todo

    \end{enumerate}


\end{document}

