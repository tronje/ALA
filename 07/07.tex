\newcommand{\authorinfotitle}{Jonathan Siems, 6533519, Gruppe 12\\ Jan-Thomas Riemenschneider, 6524390, Gruppe 12 \\ Tronje Krabbe, 6435002, Gruppe 9}
\newcommand{\authorinfo}{Jonathan Siems, Tronje Krabbe, Jan-Thomas Riemenschneider}

\newcommand{\qed}{\square}
\newcommand{\limp}{\underset{n \to \infty}{\lim}}
\newcommand{\limn}{\underset{n \to - \infty}{\lim}}
\newcommand{\todo}{\textbf{\textcolor{red}{TODO}}}
\newcommand{\todaynum}{\the\day.\the\month.\the\year}
\newcommand{\abgabe}{\textbf{\textcolor{red}{ABGABEDATUM}}}
\newcommand{\titleinfo}{ALA 07 29.05.2014}

\documentclass[a4paper,11pt]{article}
\usepackage[a4paper]{geometry}
\usepackage[german,ngerman]{babel}
\usepackage[utf8]{inputenc}
\usepackage[T1]{fontenc}
\usepackage{amsmath,amssymb,amstext}
\usepackage{mathtools}
\usepackage{enumerate}
\usepackage{breqn}
\usepackage{fancyhdr}
\usepackage{multicol}
\usepackage{color}
\usepackage{microtype}
\usepackage{booktabs}
\usepackage{lmodern}
\usepackage{tikz}
\usepackage{pgfplots}
\usepackage{scrdate}
\usetikzlibrary{calc}

\title{\titleinfo}
\author{\authorinfotitle}
\date{\today}

\pagestyle{fancy}
\fancyhf{}
\fancyhead[R]{\authorinfo}
\fancyhead[L]{ALA Hausaufgaben}
\fancyfoot[C]{\thepage}


\begin{document}
\maketitle
    \begin{enumerate}
        % Aufgabe 1
        \item[\textbf{1.}]
        \begin{enumerate}
            \item[a)]
                \begin{align*}
                    \begin{tikzpicture}
                        \begin{axis}[
                            ymin=0,ymax=10,
                            xmin=-4,xmax=7,
                            x=1cm, y=0.5cm,
                            axis x line=middle,
                            axis y line=middle,
                            axis line style=->,
                            xlabel={$x$},
                            ylabel={$f(x)$},
                            ]
                            \addplot[thick, no marks, black, -] expression[samples=100]{e^(-x)};
                        \end{axis}
                    \end{tikzpicture}
                \end{align*}
                Die $e$-Funktion besitzt keinen Wendepunkt, genauso wenig wie $e^{-x}$.

                \begin{align*}
                    \begin{tikzpicture}
                        \begin{axis}[
                            ymin=0,ymax=1.5,
                            xmin=0,xmax=20,
                            x=0.5cm, y=2cm,
                            axis x line=middle,
                            axis y line=middle,
                            axis line style=->,
                            xlabel={$x$},
                            ylabel={$g(x)$},
                            ]
                            \addplot[thick, no marks, black, -] expression[domain=0:20,samples=100]{1/(1+x)};
                        \end{axis}
                    \end{tikzpicture}
                \end{align*}
                Wendepunktberechnung:
                \begin{align*}
                    g(x) &= \frac{1}{1+x}\\
                    g'(x) &= - \frac{1}{(1+x)^2}\\
                    g''(x) &= \frac{2}{(1+x)^3}\\
                    \\
                    0 &= \frac{2}{(1+x)^3}\\
                    \Leftrightarrow 0 &= 2
                \end{align*}
                Auch $g(x)$ hat keinen Wendepunkt.

                \begin{align*}
                    \begin{tikzpicture}
                        \begin{axis}[
                            ymin=0,ymax=1.5,
                            xmin=0,xmax=20,
                            x=0.5cm, y=2cm,
                            axis x line=middle,
                            axis y line=middle,
                            axis line style=->,
                            xlabel={$x$},
                            ylabel={$h(x)$},
                            ]
                            \addplot[thick, no marks, black, -] expression[domain=0:20,samples=100]{1/(1+x^2)};
                        \end{axis}
                        \node[fill, inner sep=2pt, circle, label=above right:{\small Wendepunkt}] at (0.253535,1.45) {};
                    \end{tikzpicture}
                \end{align*}

                \begin{align*}
                    h(x) &= \frac{1}{1+x^2} \\
                    h'(x) &= - \frac{1}{(1+x^2)^2} \cdot 2x = - \frac{2x}{(1+x^2)^2}\\
                    h''(x) &\overset{*}{=} \frac{6x^2-2}{(1+x^2)^3}
                \end{align*}
                * Die Anwendungen der Quotienten- und Kettenregel wurden hier nicht ausgeführt.
                \begin{align*}
                    0 &= \frac{6x^2-2}{(1+x^2)^3} \\
                    \Leftrightarrow 0 &= 6x^2-2 \\
                    \Leftrightarrow \sqrt{\frac{1}{3}} = x
                \end{align*}
                Der Wendepunkt von $h$ liegt also bei $x=\sqrt{\frac{1}{3}}$ und $h(\sqrt{\frac{1}{3}})=\frac{1}{1+\frac{1}{3}}=\frac{3}{4}$.
            \item[b)]
                \subitem (i)
                    \begin{align*}
                        \int_0^\infty e^{-x} = \underset{b \to \infty}{\lim} [-e^{-x}]_0^b = \underset{b \to \infty}{\lim} -e^{-b} +1 = 1
                    \end{align*}
                    Der Flächeninhalt ist also 1.

                \subitem (ii)
                    \begin{align*}
                        \int_0^\infty \frac{1}{1+x} = [\log(1+x)]_0^\infty = \underset{b \to \infty}{\lim} \log(b+1) - \log(1) = \infty
                    \end{align*}
                    Der Flächeninhalt ist also unendlich groß.

                \subitem (iii)
                    \begin{align*}
                        \int_0^\infty \frac{1}{1+x^2} = [\tan^{-1}(x)]_0^\infty = \underset{b \to \infty}{\lim} \tan^{-1}(b) - \tan^{-1}(0)
                        = \frac{\pi}{2}
                    \end{align*}
            \item[c)]

            \begin{align*}
               & \text{Skizze:} \\ \\
                    &\begin{tikzpicture}
                    \begin{axis}[
                        ymin=0,ymax=5,
                        xmin=-1.2,xmax=1.2,
                        x=2.5cm, y=0.8cm,
                        axis x line=middle,
                        axis y line=middle,
                        axis line style=->,
                        xlabel={$x$},
                        ylabel={$y$},
                        ]
                        \addplot[very thick, no marks, black, -] expression[domain=-1:1,samples=100]{1/sqrt(1-x^2)};
                    \end{axis}
                \end{tikzpicture} \\ \\
                    &\text{Für ein bestimmtes Integral berechnen wir die Fläche zwischen x und dem Graphen:} \\
                    & \int_{-1}^1  \frac{1}{\sqrt{1-x^2}} =  \Big[ sin^{-1}(x) \Big]_{-1}^{ \ 1} = sin^{-1}(1) - sin^{-1}(-1) \\ \\ &\approx 3.1415926535897932384626433832795028841971693993751058
            \end{align*}
        \end{enumerate}


        % Aufgabe 2
        \item[\textbf{2.}]
            \subitem a)
                \begin{align*}
                    &\sum_{i=1}^\infty \frac{i}{2^i}\\
                    &\overset{*}{=} \sum_{i=0}^\infty \frac{i}{2^i}
                \end{align*}
                * Dies gilt, da $\frac{0}{2^0} = 0$.\\
                Es gelte:
                \begin{align*}
                    \underset{i \to \infty}{\lim} \sqrt[i]{\frac{i}{2^i}} < 1
                \end{align*}
                Da $\sqrt[i]{i} \to 1$ für $i \to \infty$ und $2^i \geq 1$, ist diese Aussage korrekt. Somit konvergiert die Reihe.
            \subitem b)
                gaaaaay
        % Aufgabe 3
        \item[\textbf{3.}]
	\todo

        % Aufgabe 4
        \item[\textbf{4.}]
	\todo

    \end{enumerate}


\end{document}

