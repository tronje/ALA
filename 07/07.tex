\newcommand{\authorinfotitle}{Jonathan Siems, 6533519, Gruppe 12\\ Jan-Thomas Riemenschneider, 6524390, Gruppe 12 \\ Tronje Krabbe, 6435002, Gruppe 9}
\newcommand{\authorinfo}{Jonathan Siems, Tronje Krabbe, Jan-Thomas Riemenschneider}

\newcommand{\qed}{\square}
\newcommand{\limp}{\underset{n \to \infty}{\lim}}
\newcommand{\limn}{\underset{n \to - \infty}{\lim}}
\newcommand{\todo}{\textbf{\textcolor{red}{TODO}}}
\newcommand{\todaynum}{\the\day.\the\month.\the\year}
\newcommand{\abgabe}{\textbf{\textcolor{red}{ABGABEDATUM}}}
\newcommand{\titleinfo}{ALA 07 29.05.2014}

\documentclass[a4paper,11pt]{article}
\usepackage[a4paper]{geometry}
\usepackage[german,ngerman]{babel}
\usepackage[utf8]{inputenc}
\usepackage[T1]{fontenc}
\usepackage{amsmath,amssymb,amstext}
\usepackage{mathtools}
\usepackage{enumerate}
\usepackage{breqn}
\usepackage{fancyhdr}
\usepackage{multicol}
\usepackage{color}
\usepackage{microtype}
\usepackage{booktabs}
\usepackage{lmodern}
\usepackage{tikz}
\usepackage{pgfplots}
\usepackage{scrdate}
\usetikzlibrary{calc}

\title{\titleinfo}
\author{\authorinfotitle}
\date{\today}

\pagestyle{fancy}
\fancyhf{}
\fancyhead[R]{\authorinfo}
\fancyhead[L]{ALA Hausaufgaben}
\fancyfoot[C]{\thepage}


\begin{document}
\maketitle
    \begin{enumerate}
        % Aufgabe 1
        \item[\textbf{1.}]
        \begin{enumerate}
            \item[a)]
                \begin{align*}
                    \begin{tikzpicture}
                        \begin{axis}[
                            ymin=0,ymax=10,
                            xmin=-4,xmax=7,
                            x=1cm, y=0.5cm,
                            axis x line=middle,
                            axis y line=middle,
                            axis line style=->,
                            xlabel={$x$},
                            ylabel={$f(x)$},
                            ]
                            \addplot[thick, no marks, black, -] expression[samples=100]{e^(-x)};
                        \end{axis}
                    \end{tikzpicture}
                \end{align*}
                Die $e$-Funktion besitzt keinen Wendepunkt, genauso wenig wie $e^{-x}$.

                \begin{align*}
                    \begin{tikzpicture}
                        \begin{axis}[
                            ymin=0,ymax=1.5,
                            xmin=0,xmax=50,
                            x=0.2cm, y=2cm,
                            axis x line=middle,
                            axis y line=middle,
                            axis line style=->,
                            xlabel={$x$},
                            ylabel={$g(x)$},
                            ]
                            \addplot[thick, no marks, black, -] expression[domain=0:100,samples=100]{1/(1+x)};
                        \end{axis}
                    \end{tikzpicture}
                \end{align*}
                Wendepunktberechnung:
                \begin{align*}

                \end{align*}

                \begin{align*}
                h(x) &= \frac{1}{1+x^2} \\
                h'(x) &= \frac{0 \cdot (1+x^2)-2x \cdot 1}{(1+x^2)^2} \Rightarrow \frac{-2x}{(1+x^2)} \\
                h''(x) &= \frac{-2 \cdot(1+x^2)^2 - 2(1+x^2) \cdot -4x^2}{(1+x^2)^4} \quad = \quad \frac{-2(1+x^2)-(-8x^2)}{(1+x^2)^3} \\
                &= \frac{-2 -2x^2 + 8x^2}{1+x^2)^3} \quad = \quad \frac{6x^2-2}{(1+x^2)^3}\\ \\
                % \text{Mit Hilfe }& \text{der PQ-Formel lösen wir nach x auf:} \\
                % p &= \frac{0}{2} \qquad q = -\frac{1}{3} \\ \\
                % x_{1|2} &= \frac{0}{2} \pm \sqrt{\left(\frac{0}{2}\right)^2 -(-\frac{1}{3})} \\ \\
                % x_1 &\approx 0,577 \\
                % x_2 &\approx - 0,577 
                \end{align*}

            \item[b)]

            \item[c)]

            \begin{align*}
               & \text{Skizze:} \\ \\
                    &/\begin{tikzpicture}
                    \begin{axis}[
                        ymin=0,ymax=5,
                        xmin=-1.2,xmax=1.2,
                        x=2.5cm, y=0.8cm,
                        axis x line=middle,
                        axis y line=middle,
                        axis line style=->,
                        xlabel={$x$},
                        ylabel={$y$},
                        ]
                        \addplot[very thick, no marks, black, -] expression[domain=-1:1,samples=100]{1/sqrt(1-x^2)};
                    \end{axis}
                \end{tikzpicture} \\ \\
                         &\text{Für ein bestimmtes Integral berechnen wir die Fläche zwischen x und dem Graphen:} \\
                        & \int_{-1}^1  \frac{1}{\sqrt{1-x^2}} =  \Big[ sin^{-1}(x) \Big]_{-1}^{ \ 1} = sin^{-1}(1) - sin^{-1}(-1) \\ \\ &\approx 3.1415926535897932384626433832795028841971693993751058
                \end{align*}



        \end{enumerate}


        % Aufgabe 2
        \item[\textbf{2.}]
            \todo

        % Aufgabe 3
        \item[\textbf{3.}]
	\todo

        % Aufgabe 4
        \item[\textbf{4.}]
	\todo

    \end{enumerate}


\end{document}

