\newcommand{\authorinfotitle}{Jonathan Siems, 6533519, Gruppe 12\\ Jan-Thomas Riemenschneider, 6524390, Gruppe 12 \\ Tronje Krabbe, 6435002, Gruppe 9}
\newcommand{\authorinfo}{Jonathan Siems, Tronje Krabbe, Jan-Thomas Riemenschneider}

\newcommand{\qed}{\square}
\newcommand{\limp}{\underset{n \to \infty}{\lim}}
\newcommand{\limn}{\underset{n \to - \infty}{\lim}}
\newcommand{\todo}{\textbf{\textcolor{red}{TODO}}}
\newcommand{\todaynum}{\the\day.\the\month.\the\year}
\newcommand{\abgabe}{\textbf{\textcolor{red}{ABGABEDATUM}}}
\newcommand{\titleinfo}{ALA 07 29.05.2014}

\documentclass[a4paper,11pt]{article}
\usepackage[a4paper]{geometry}
\usepackage[german,ngerman]{babel}
\usepackage[utf8]{inputenc}
\usepackage[T1]{fontenc}
\usepackage{amsmath,amssymb,amstext}
\usepackage{mathtools}
\usepackage{enumerate}
\usepackage{breqn}
\usepackage{fancyhdr}
\usepackage{multicol}
\usepackage{color}
\usepackage{microtype}
\usepackage{booktabs}
\usepackage{lmodern}
\usepackage{tikz}
\usepackage{pgfplots}
\usepackage{scrdate}
\usepackage{ulem}
\usetikzlibrary{calc}

\title{\titleinfo}
\author{\authorinfotitle}
\date{\today}

\pagestyle{fancy}
\fancyhf{}
\fancyhead[R]{\authorinfo}
\fancyhead[L]{ALA Hausaufgaben}
\fancyfoot[C]{\thepage}


\begin{document}
\maketitle
    \begin{enumerate}
        % Aufgabe 1
        \item[\textbf{1.}]
        \begin{enumerate}
            \item[a)]
                \begin{align*}
                    \begin{tikzpicture}
                        \begin{axis}[
                            ymin=0,ymax=10,
                            xmin=-4,xmax=7,
                            x=1cm, y=0.5cm,
                            axis x line=middle,
                            axis y line=middle,
                            axis line style=->,
                            xlabel={$x$},
                            ylabel={$f(x)$},
                            ]
                            \addplot[thick, no marks, black, -] expression[samples=100]{e^(-x)};
                        \end{axis}
                    \end{tikzpicture}
                \end{align*}
                Die $e$-Funktion besitzt keinen Wendepunkt, genauso wenig wie $e^{-x}$.

                \begin{align*}
                    \begin{tikzpicture}
                        \begin{axis}[
                            ymin=0,ymax=1.5,
                            xmin=0,xmax=20,
                            x=0.5cm, y=2cm,
                            axis x line=middle,
                            axis y line=middle,
                            axis line style=->,
                            xlabel={$x$},
                            ylabel={$g(x)$},
                            ]
                            \addplot[thick, no marks, black, -] expression[domain=0:20,samples=100]{1/(1+x)};
                        \end{axis}
                    \end{tikzpicture}
                \end{align*}
                Wendepunktberechnung:
                \begin{align*}
                    g(x) &= \frac{1}{1+x}\\
                    g'(x) &= - \frac{1}{(1+x)^2}\\
                    g''(x) &= \frac{2}{(1+x)^3}\\
                    \\
                    0 &= \frac{2}{(1+x)^3}\\
                    \Leftrightarrow 0 &= 2
                \end{align*}
                Auch $g(x)$ hat keinen Wendepunkt.

                \begin{align*}
                    \begin{tikzpicture}
                        \begin{axis}[
                            ymin=0,ymax=1.5,
                            xmin=0,xmax=20,
                            x=0.5cm, y=2cm,
                            axis x line=middle,
                            axis y line=middle,
                            axis line style=->,
                            xlabel={$x$},
                            ylabel={$h(x)$},
                            ]
                            \addplot[thick, no marks, black, -] expression[domain=0:20,samples=100]{1/(1+x^2)};
                        \end{axis}
                        \node[fill, inner sep=2pt, circle, label=above right:{\small Wendepunkt}] at (0.253535,1.45) {};
                    \end{tikzpicture}
                \end{align*}

                \begin{align*}
                    h(x) &= \frac{1}{1+x^2} \\
                    h'(x) &= - \frac{1}{(1+x^2)^2} \cdot 2x = - \frac{2x}{(1+x^2)^2}\\
                    h''(x) &\overset{*}{=} \frac{6x^2-2}{(1+x^2)^3}
                \end{align*}
                * Die Anwendungen der Quotienten- und Kettenregel wurden hier nicht ausgeführt.
                \begin{align*}
                    0 &= \frac{6x^2-2}{(1+x^2)^3} \\
                    \Leftrightarrow 0 &= 6x^2-2 \\
                    \Leftrightarrow \sqrt{\frac{1}{3}} = x
                \end{align*}
                Der Wendepunkt von $h$ liegt also bei $x=\sqrt{\frac{1}{3}}$ und $h(\sqrt{\frac{1}{3}})=\frac{1}{1+\frac{1}{3}}=\frac{3}{4}$.
            \item[b)]
                \subitem (i)
                    \begin{align*}
                        \int_0^\infty e^{-x} = \underset{b \to \infty}{\lim} [-e^{-x}]_0^b = \underset{b \to \infty}{\lim} -e^{-b} +1 = 1
                    \end{align*}
                    Der Flächeninhalt ist also 1.

                \subitem (ii)
                    \begin{align*}
                        \int_0^\infty \frac{1}{1+x} = [\log(1+x)]_0^\infty = \underset{b \to \infty}{\lim} \log(b+1) - \log(1) = \infty
                    \end{align*}
                    Der Flächeninhalt ist also unendlich groß.

                \subitem (iii)
                    \begin{align*}
                        \int_0^\infty \frac{1}{1+x^2} = [\tan^{-1}(x)]_0^\infty = \underset{b \to \infty}{\lim} \tan^{-1}(b) - \tan^{-1}(0)
                        = \frac{\pi}{2}
                    \end{align*}
            \item[c)]

            \begin{align*}
               & \text{Skizze:} \\ \\
                    &\begin{tikzpicture}
                    \begin{axis}[
                        ymin=0,ymax=5,
                        xmin=-1.2,xmax=1.2,
                        x=2.5cm, y=0.8cm,
                        axis x line=middle,
                        axis y line=middle,
                        axis line style=->,
                        xlabel={$x$},
                        ylabel={$f(x)$},
                        ]
                        \addplot[very thick, no marks, black, -] expression[domain=-1:1,samples=100]{1/sqrt(1-x^2)};
                    \end{axis}
                \end{tikzpicture} \\ \\
                    &\text{Für ein bestimmtes Integral berechnen wir die Fläche zwischen x und dem Graphen:} \\
                    & \int_{-1}^1  \frac{1}{\sqrt{1-x^2}} =  \Big[ sin^{-1}(x) \Big]_{-1}^{ \ 1} = sin^{-1}(1) - sin^{-1}(-1) \\ \\ &\approx 3.1415926535897932384626433832795028841971693993751058
            \end{align*}
        \end{enumerate}


        % Aufgabe 2
        \item[\textbf{2.}]
            \begin{enumerate}
            \item[a)]
                \begin{align*}
                    &\sum_{i=1}^\infty \frac{i}{2^i}\\
                    &\overset{*}{=} \sum_{i=0}^\infty \frac{i}{2^i}
                \end{align*}
                * Dies gilt, da $\frac{0}{2^0} = 0$.\\
                Es gelte:
                \begin{align*}
                    \underset{i \to \infty}{\lim} \sqrt[i]{\frac{i}{2^i}} < 1
                \end{align*}
                Da $\sqrt[i]{i} \to 1$ für $i \to \infty$ und $2^i \geq 1$, ist diese Aussage korrekt. Somit konvergiert die Reihe.
            \item[b)]
                Wir betrachten den folgenden Grenzwert:
                \begin{align*}
                    &\underset{i \to \infty}{\lim} \Big| \frac{\frac{(-1)^{i+1}\cdot (i+1)!}{(i+1)^{i+1}}}{\frac{(-1)^i \cdot i!}{i^i}} \Big|\\
                    &= \underset{i \to \infty}{\lim} \Big| \frac{\frac{(-1)^{i+1}\cdot i!}{(i+1)^{i}}}{\frac{(-1)^i \cdot i!}{i^i}} \Big|\\
                    &= \underset{i \to \infty}{\lim} \Big| \frac{(-1)^{i+1}\cdot i! \cdot i^i}{(i+1)^i (-1)^i \cdot i!} \Big|\\
                    &= \underset{i \to \infty}{\lim} \Big| - \frac{i^i}{(i+1)^i} \Big|\\
                    &= \underset{i \to \infty}{\lim} \frac{i^i}{(i+1)^i}\\
                    &\overset{*}{=} \frac{1}{e}
                \end{align*}
                * Weil das ja klar ist.\\
                Da $\frac{1}{e} < 1$ ist die Konvergenz nachgewiesen.
            \end{enumerate}
        % Aufgabe 3
        \item[\textbf{3.}]
            Für $\sum_{i=0}^\infty i^22^ix^i$ soll der Konvergenzradius ermittelt werden.
            \begin{enumerate}
            \item[a)]
                Mithilfe der Limes-Version des Quotientenkriteriums:\\
                Wir betrachten den folgenden Grenzwert. Ist dieser kleiner als 1, so liegt Konvergenz vor. Ist er größer, so divergiert
                die Reihe. Wir wählen ein beliebiges $x \in \mathbb{R}$; in der folgenden Rechnung ist $x$ also fest gewählt.
                \begin{align*}
                    &\underset{i \to \infty}{\lim} \Big|\frac{(i+1)^22^{i+1}x^{i+1}}{i^22^ix^i}\Big|\\
                    &= \underset{i \to \infty}{\lim} \Big|\frac{(i+1)^2}{i^2}\cdot 2x \Big|\\
                    &= \underset{i \to \infty}{\lim}\left( \Big|\frac{(i+1)^2}{i^2}\Big| \cdot |2x| \right)\\
                    &\overset{*}{=} 2|x| \cdot \underset{i \to \infty}{\lim} \Big|\frac{(i+1)^2}{i^2}\Big|\\
                    &= 2|x|
                \end{align*}
                * An dieser Stelle wurde genutzt, dass $x$ fest gewählt wurde.
                Ausserdem gilt:
                \begin{align*}
                    2|x| \Leftrightarrow |x| < \frac{1}{2}
                \end{align*}
                Daraus folgt: $R = \frac{1}{2}$
            \item[b)]
                Mithilfe der Limes-Version des Wurzelkriteriums:\\
                Wir betrachten abermals einen Grenzwert, und es gilt abermals, dass die Reihe konvergiert, wenn der besagt Grenzwert
                kleiner als 1 ist. Es sei wieder $x \in \mathbb{R}$ beliebig, aber fest gewählt.
                \begin{align*}
                    &\underset{i \to \infty}{\lim} \sqrt[i]{|i^22^ix^i|}\\
                    &= \underset{i \to \infty}{\lim} \left(\sqrt[i]{i^2}\cdot\sqrt[i]{2^i}\cdot\sqrt[i]{x^i}\right)\\
                    &\overset{*}{=} 2x \cdot \underset{i \to \infty}{\lim} \left(\sqrt[i]{i}\cdot \sqrt[i]{i}\right)\\
                    &= 2x
                \end{align*}
                * An dieser Stelle wurde genutzt, dass $x$ fest gewählt wurde. Weiterhin gilt:
                \begin{align*}
                    2x < 1 \Leftrightarrow x < \frac{1}{2}
                \end{align*}
                Dementsprechend gilt $R = \frac{1}{2}$.
            \end{enumerate}
        % Aufgabe 4
        \item[\textbf{4.}]
	       \todo

        \item[\textbf{5.}]
            \begin{enumerate}

                \item[a)]

                    $$ \int\limits_{1}^{n+1}  \frac{1}{x} dx  \ \leq H_n \quad (n=1,2,...). \quad \Big| \quad \text{Es gilt }n \in \mathbb{N}$$

                    \begin{tikzpicture}
                    \begin{axis}[
                        ymin=0,ymax=1.5,
                        xmin=0,xmax=8,
                        x=1.5cm, y=3cm,
                        axis x line=middle,
                        axis y line=middle,
                        axis line style=->,
                        xlabel={$x$},
                        ylabel={$f(x)$},
                        ]
                        \addplot[very thick, no marks, black, -] expression[domain=0:8,samples=100]{1/x};
                    \end{axis}
                \end{tikzpicture}



                Anhand der Skizze kann man erkennen,\\ dass für einen größeren $x$-Wert der Wert für $f(x)$ abnimmt, \\
                da die Funktion für  $x \rightarrow \infty$ gegen $0$ geht. \\
                Wir erhalten für jeden Wert $\geq$ 1 den wir für $x$ einsetzen einen Wert $\leq$ 1. 

                Da wir bei der Berechnung eines bestimmten Integrals bei einem gleichen Wert für Integrationsober- und untergrenze immer $0$ als Ergebnis erhalten würden, \\ ist für die Obergrenze hier $n+1$ gewählt. \\[0.3cm]
                Bei der Integration von $\int \frac{1}{x}dx$ erhalten wir $\Big[ln(x)\Big]$,\\ für die Untergrenze also $ln(1) = 0$. \\
                Somit müssen wir für die Berechnung des Integrals $ \Big(ln(n+1) -$ \sout{$ ln(1)$}$ \Big)$ \\ lediglich den Wert der oberen Integrationsgrenze in $ln(x)$ einsetzen. \\ 
                Man sieht, dass $ln(n+1)$ im Verhältnis zu $n$ sehr langsam wächst. \\ \\
                Für jeden Wert von n gilt daher $\uuline{ln(n+1) \leq n}$ \\
                Ein Beweis ist in diesem Falle obsolet, man könne ihn aber mithilfe vollständiger Induktion erbringen.


            \end{enumerate}


            \item[\textbf{6.}]
            \todo

    \end{enumerate}


\end{document}

